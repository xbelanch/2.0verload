%PROGIT Template 
%http://github.com/progit/progit/blob/master/latex/template.tex
\documentclass[12pt, a4paper,twoside]{book}
\usepackage[
	urlbordercolor ={1 1 1},
	linkbordercolor ={1 1 1},
	citebordercolor ={1 1 1},
	urlcolor = blue,
	colorlinks = true,
	citecolor = black,
	linkcolor = black]{hyperref}
\usepackage{graphicx}
\usepackage[hyper]{apacite}
\usepackage{xltxtra}
\usepackage{fancyhdr}
\usepackage{booktabs}

\setromanfont{Baskerville}
\setmonofont{Courier}

\XeTeXlinebreaklocale{es}

\title{2.0verload}
\author{Xavier Belanche Alonso}

\makeatletter
\let\savedauthor=\@author
\let\savedtitle=\@title
\def\maxwidth{\ifdim\Gin@nat@width>\linewidth\linewidth
\else\Gin@nat@width\fi}
\makeatother

\fancypagestyle{plain}{\fancyhf{}\fancyfoot[LE,RO]{\footnotesize\textbf\thepage}}
\pagestyle{plain}

\renewcommand{\headrulewidth}{0pt}
\renewcommand{\footrulewidth}{0pt}

\newcounter{img}[chapter]
\renewcommand{\theimg}{\thechapter.\arabic{img}}
\newcommand{\img}[1]{\begin{figure}[h!]
	\refstepcounter{img}
	\label{img:\theimg}
	\centering\includegraphics[width=\maxwidth]{figures/\theimg.png}
	\textbf{\caption{#1}}
\end{figure}}

\renewcommand{\chaptermark}[1]{\markboth{\textbf{Capítulo \thechapter}\hspace*{1ex}#1}{}}
\renewcommand{\sectionmark}[1]{\markright{\textbf{Sección \thesection}\hspace*{1ex}#1}}
\newcommand{\chap}[1]{\newpage\thispagestyle{empty}\chapter{#1}\label{chap:\thechapter}}
\newcommand{\chapref}[1]{\hyperref[chap:#1]{Capítulo #1}}
\newcommand{\imgref}[1]{\hyperref[img:#1]{Figura #1}}
\newcommand{\tabref}[1]{\hyperref[tab:#1]{Tabla #1}}
\newcommand{\e}[1]{$ \times 10^{#1}$}

\linespread{1.03}

\begin{document}
\frontmatter
\maketitle
\thispagestyle{empty}
\setcounter{tocdepth}{4}
\tableofcontents\newpage\thispagestyle{empty}

\mainmatter
\fancyhf{}
\fancyhead[LO]{{\small\leftmark}}
\fancyhead[LE]{{\small\rightmark}}
\fancyhead[R]{{\small\savedauthor\hspace*{1ex}\textbf{\savedtitle}}}
\fancyfoot[LE,RO]{\small\textbf\thepage}
\pagestyle{fancy}

\chap{Notas antes de partir}

La percepción de estar inmersos en una etapa irreversible de exceso
de información no es exclusiva de nuestra época. A finales del
siglo XVII, Adrien Baillet, lector apasionado y biógrafo de René
Descartes, escribió:

\begin{quote}
that the multitude of books which grows every day in a prodigious
fashion will make the following centuries fall into a state as
barbarous as that of the centuries that followed the fall of the
Roman Empire.

\end{quote}
Debido a la incapacidad de absorber y evaluar el flujo continuo de
información, Adrien Baillet alertó de un futuro confuso, una espesa
niebla que nos impediría discernir lo relevante de lo superfluo;
empujaría a la civilización occidental a una nueva etapa oscura de
la cultura, esta vez no por ausencia cuantitativa de información,
si no cualitativa. El anticipo de la banalización de la cultura y,
finalmente, el regreso a una suerte de Edad Media donde los
ciudadanos se ven anestesiados por el enorme volumen de
información.

Si la cuestión del exceso de información se limitó a las
reflexiones de la élite cultural del siglo XVI y XVII, en la
actualidad es un fenómeno que afecta a todos por igual. Ann Blair%
\footnote{Blair, Ann. (2003).
\emph{Coping with Information Overload in Early Modern Europe}},
en un estudio sobre cómo hacer frente al exceso de información
concluye que

\begin{quote}
I don't have any particular remedy to offer, other than to comment
that even though the problem seems dire, it's not all that new.

\end{quote}
El exceso de información es un relato recurrente; proyecta una
sombra que distorsiona las actividades relacionadas con la
información, es decir, su adquisición, búsqueda, clasificación,
categorización, recuperación y preservación. Gestionar el exceso de
información es objeto de análisis y advertencias. Un reciente
estudio%
\footnote{Wilson, (2005).
\emph{\href{http://www.newscientist.com/article/mg18624973.400}{Info-overload harms concentration more than marijuana}}}
advierte que la \emph{infomanía} o la \emph{infobesidad} tiene
peores efectos que el consumo de marihuana. La \emph{infoxicación}
provoca estrés, un penosa sensación de improductividad; se extiende
la multitarea como metodología única de trabajo y nos acaba
castigando al difuminar la frontera entre la parcela íntima,
privada y profesional. De un tiempo a esta parte los libros de
autoayuda en el terreno profesional han invadido las estanterías de
librerías. De un tiempo a esta parte, los blogs sobre productividad
personal y excelencia laboral proliferan y aumentan en número.
Consejos, recetas prefabricadas, opiniones, estudios, tutoriales,
experiencias personales alrededor de un mismo eje: controla el
exceso de información antes que sucumbas a él.

\begin{quote}
E-mail traffic has increased, computers and other devices have
profilerated, new acronyms and technology terms have invaded our
speech, and many people can sense that there's a problem. It's all
too much%
\footnote{Hurst, Mark. (2007). \emph{Bit Literacy}. Good Experience Press.}

\end{quote}
Se asocia el exceso de información al déficit de atención%
\footnote{Hallowell, E. M. (2005).
\emph{\href{http://tr.im/wSsS}{Overloaded circuits: Why smart people underperform}}},
a la toma equivocada de decisiones%
\footnote{Malhotra, Naresh K. (1984).
\emph{\href{http://www.jstor.org/pss/2488913}{Reflections on the Information Overload Paradigm in Consumer Decision Making}}}
y como al tiempo que tardamos en tomarlas%
\footnote{Jacoby, J. (1984),
\emph{\href{http://www.jstor.org/pss/2488912}{Perspectives on Information Overload}}}.
Nos ahogamos en el detalle y perdemos de vista la perspectiva
general de las situación%
\footnote{Schneider, Ursula. (2002).
\emph{\href{http://www.jucs.org/jucs_8_5/the_knowledge_attention_gap/Schneider_U.html}{The Knowledge-Attention-Gap: Do We Underestimate The Problem Of Information Overload In Knowledge Management?}}}.
La economía de la atención emerge del océano informativo. No hay
tiempo para una lectura que no sea en diagonal; no hay espacios de
reflexión, pero sí de inmediatez, de conversación, de una suerte de
\emph{oralidad} reinventada por los nuevos medios de comunicación.

El relato de la abundancia de información adquiere un significado
especial gracias a las nuevas tecnologías de la web. Convivimos con
la promesa de una integridad inquebrantable del circuito del
tratamiento de la información. La irreversible digitalización de la
información personal se la acompaña de buscadores más inteligentes,
más precisos en los resultados, más rápidos de respuesta a las
consultas; buscar información nunca nos ha parecido tan fácil. Los
márgenes del contenedor en el que guardamos la información, que
seleccionamos y categorizamos son igualmente difusos: los creemos
ilimitados. Un relato en que los diferentes servicios ubicados bajo
ese indefinible paraguas tecnológico, pero no menos social,
conocido popularmente por \emph{Web 2.0}, han construido y forjado
hábilmente en el imaginario colectivo.

Parafraseando a Roland Barthes%
\footnote{Barthes, Roland. 1980. \emph{La cámara lúcida}. Paidós},
este libro defraudará a técnicos, \emph{geeks} como a quienes
buscan o esperan descubrir una pésima liturgia de autoayuda digital
o un interminable listado de servicios de la ubicua y asfixiante
etiqueta \emph{2.0}. Por el contrario, este libro habla de cómo el
fenómeno de la \emph{Web 2.0} ha reinventado las actividades
asociadas a la información personal y, a un mismo tiempo, dice
adiós a la metáfora del escritorio individual a favor de un
escritorio colectivo. Una nueva orografía, por tanto, del exceso de
información personal disfrazada ahora de la \emph{conversación}.
Como el cartógrafo de Vermeer, sujetando el instrumento de medida
mientras medita y reflexiona sobre los límites de la
representación, cada capítulo del libro corresponde a una
descripción incierta y a veces equivocada de una región de esta
\emph{conversación} bajo el prisma poliédrico de la web actual, eco
y reflejo de aquella ciudad invisible de Octavia

\begin{quote}
Suspendida en el abismo, la vida de los habitantes de Octavia es
menos incierta que en otras ciudades. Sabe que la resistencia de la
red tiene un límite%
\footnote{Italo Calvino, \emph{Las ciudades invisibles}}

\end{quote}
\chap{Personal Information Glut}

La marca \emph{2.0} ha impuesto la dictadura del
\emph{antes y el después}. Una división irreconciliable entre lo
viejo -pesado, ineficiente, gris e individual- y lo nuevo -ágil,
competente, colorido y colectivo-.

Generalmente se identifica la Web 2.0 como una segunda generación
de la web%
\footnote{Blair, Ann. (2003).
\emph{Coping with Information Overload in Early Modern Europe}},
en contraposición al modelo anterior de la burbuja punto com, en la
que se estimula la participación social en la creación y
distribución de contenidos. Bajo el signo 2.0 han aparecido
servicios como YouTube, Flickr, Twitter o Facebook, hoy en día tan
universalmente populares como controvertidos; se habla y se
utilizan nuevas herramientas de comunicación -wikis, blogging,
microblogging, \emph{vblogging}, podcasting,\ldots{}- y la
sofisticación de otras -videochats, mensajería instantánea-. La Web
2.0 es también la web de la lectura-escritura. De la reinvención de
tecnologías como el caso del denostado Javascript, que parecían
estar abocadas a una severa marginación. Del redescubrimiento de
otras (Python, Ruby). De la necesaria difusión y aceptación de los
estándares de la Web. De un nuevo pulso entre navegadores web
(Mozilla Firefox, Internet Explorer, Google Chrome, Opera, Safari)
cuando a principios del 2000 todo parecía indicar una hegemonía
indiscutible -ética y tecnológicamente lamentable- de Internet
Explorer. De la intercomunicación entre dispositivos físicos y
virtuales (códigos de barra bidimensionales, realidad aumentada).
De la W2W, ``Web to wealth''%
\footnote{Wilson, (2005).
\emph{\href{http://www.newscientist.com/article/mg18624973.400}{Info-overload harms concentration more than marijuana}}}.
De una renovada agitación en el ámbito de los negocios
\emph{punto com}%
\footnote{Hurst, Mark. (2007). \emph{Bit Literacy}. Good Experience Press.}.
Finalmente, de la Web 2.0 como peldaño o bisagra a la ansiada e
imaginada web semántica.

Pero la Web 2.0 no excluye sombras. Se ha visto -y se sigue viendo-
la Web 2.0 como una respuesta por parte de la industria contra los
sistemas P2P. Su misión es aniquilar el aspecto descentralizado y
distribuido de internet. Los usuarios que comparten y distribuyen
información a través de clientes P2P respiran el aire de la
ausencia de jerarquías, de un control centralizado. La Web 2.0 se
vende como marca regulada de una web distribuida. Puro espejismo.
Se aboga antes por soluciones controladas de la circulación de
información y, por consiguiente, de su control y censura. Es más
fuerte la presión y persecución judicial de los grupos P2P que no a
los actuales servicios 2.0.

\begin{quote}
Thus Web 2.0 is not to be thought of as a second-generation of
either the technical or social development of the internet, but
rather as the second wave of capitalist enclosure of the
Information Commons%
\footnote{Hallowell, E. M. (2005).
\emph{\href{http://tr.im/wSsS}{Overloaded circuits: Why smart people underperform}}}

\end{quote}
Si la Web 2.0 se ha edificado en el relato de una conversación
colectiva%
\footnote{Malhotra, Naresh K. (1984).
\emph{\href{http://www.jstor.org/pss/2488913}{Reflections on the Information Overload Paradigm in Consumer Decision Making}}}:
McLellan, 2008 donde todas las voces tienen algo qué decir
(\emph{¿Qué estás haciendo en este momento?}), ¿cómo se representa
el espacio de la información personal en esta fiesta permanente de
cotorras y papagayos? La Web 2.0 se ha emancipado de ser una
esclava de las metáforas del escritorio personal para construir su
propia representación, obsesionada en la edificación de una moderna
tradición oral. Es una disrupción de la palabra escrita hacia la
escritura en voz alta. Todo mensaje adopta la forma de charla%
\footnote{Jacoby, J. (1984),
\emph{\href{http://www.jstor.org/pss/2488912}{Perspectives on Information Overload}}}
luego la rápidez en la que circula la información es infinitamente
mayor. No leemos, hablamos leyendo. Escribimos en voz alta a través
de los SMS, la mensajería instantánea, el microblogging, el
\emph{vblogging}.

\begin{quote}
Intranets naturally tend to route around boredom. The best are
built bottom-up by engaged individuals cooperating to construct
something far more valuable: an intranetworked corporate
conversation%
\footnote{Schneider, Ursula. (2002).
\emph{\href{http://www.jucs.org/jucs_8_5/the_knowledge_attention_gap/Schneider_U.html}{The Knowledge-Attention-Gap: Do We Underestimate The Problem Of Information Overload In Knowledge Management?}}}

\end{quote}
La Web 2.0 es una interminable proliferación de servicios, de
tecnologías que celebran la \emph{industria de la conversación}.
Cada poco tiempo aparece un nuevo servicio, una nueva estrategia,
una vuelta de tuerca que intenta hacerse un hueco en la economía de
la atención. Ya no sólo contamos con un exceso de información o,
mejor dicho, de voces, sino también de servicios de publicación de
información, privada o pública. Además hay que sumar a este
escenario una miríada de dispositivos -\emph{gadgets}- que conectan
sin interrupción con la conversación. Dispositivos que tienen en
común la movilidad -pequeños, ligeros y fuertemente ergonómicos-,
la interoperabilidad (necesaria, aún levantando estúpidas barreras
por parte de los fabricantes) y la conectividad. El primer síntoma
de este exceso no se limita sólo a nuestras actividades personales
de carácter inmediato -buscar y guardar información- y que luego
hablaremos con más detalle. Se rompe definitivamente la frontera
física del espacio personal de la información; ésta ya no se limita
a las cuatro paredes de nuestro despacho o estudio. El escritorio
es una pieza más de un puzzle digital, difuso en cuanto a que el
número de dispositivos con el que interactuamos -el ordenador de
casa, el portátil, el ordenador del trabajo, el móvil personal, el
móvil profesional,\ldots{}- como los servicios a los que delegamos
la información es cada vez mayor. Un espacio personal de
información flotando en una \emph{nube de ordenadores}%
\footnote{Barthes, Roland. 1980. \emph{La cámara lúcida}. Paidós}
promete anular el problema de la fragmentación de la información
repartida por diferentes dispositivos. Se predice la defunción del
escritorio tal como lo conocemos%
\footnote{Italo Calvino, \emph{Las ciudades invisibles}};
fallecimiento que no responderá a una disrupción violenta, sino a
una transición lenta de un modelo de comunicación caduco -pero
todavía mayoritario en organizaciones privadas y públicas- y
aquélla actual que basa enteramente su estrategia en la separación
definitiva entre la información -ubicua, persistente- y los
sistemas o tecnologías que actúan de bisagra entre el modo
\emph{off-line} y el acceso a la \emph{nube} de la información.

\begin{center}\rule{3in}{0.4pt}\end{center}

¿En qué medida los nuevos servicios de la llamada Web 2.0
reinventan las actividades de buscar y guardar la información
personal? ¿Se cumple por fin la promesa de una realidad ideal en la
que gracias a esta desaparición del escritorio se resuelven
definitivamente los problemas la gestión de la información
personal?

Según la Wikipedia%
\footnote{\href{http://en.wikipedia.org/wiki/Personal_information_management}{http://en.wikipedia.org/wiki/Personal\_information\_management}},
\emph{Personal Information Managament} (PIM) es el espacio personal
que engloba toda aquella actividad relacionada con la adquisición,
búsqueda, organización, recuperación, mantenimiento y uso
significativo de la información que persigue completar una
determinada tarea. Busco el teléfono de un restaurante para
reservar mesa; recupero mensajes antiguos de un cliente donde
recuerdo que en uno de ellos hablaba de un buen restaurante y así
organizar una cena de trabajo en un lugar donde se sienta cómodo.
Belloti et al. (2002) describe el PIM como
\emph{la organización de la información en base a la categorización, ubicación o adorno de la misma con el fin de facilitar su recuperación cuando sea necesaria},
o la de Barreau (1995) como
\emph{las estrategias en la adquisición, organización y guardar la información; las reglas y mecanismos para su recuperación}.

La gestión de información personal plantea una elección entre
aquello que consideramos objeto de conocimiento, que pueda
resultarnos útil, y la información que consideramos ruido y que
molesta debido a su ubicuidad. Decisión no carente de tensión:
¿dejaré escapar el ruido?, ¿y si me resulta de utilidad más
adelante? La percepción de qué consideramos información útil o
ruido puede verse alterada en un futuro. Un marcador web que hoy no
nos resulta importante, sí lo puede ser pasado un tiempo. En esta
dirección, una mayoría de servicios actuales de la web comparten la
intención de relajar esta tensión, aligerar el peso de decidir qué
hacer con el exceso de flujo de información que vamos recibiendo a
diario.

\section{Buscar, Guardar}

\begin{quote}
¡Oh! ¡el mundo offline que tanto odio! ¡Qué tedio tener que
rebuscar papeles en mi casa! ¡Qué manera de perder tiempo!%
\footnote{minid.net (2009). \emph{En casa de herrero, cuchillo de palo}
\href{http://www.minid.net/2009/06/17/en-casa-de-herrero-cuchillo-de-palo/}{http://www.minid.net/2009/06/17/en-casa-de-herrero-cuchillo-de-palo/}}

\end{quote}
Buscar y guardar son actividades atrapadas en el área de la
inmediatez. Responden a necesidades urgentes, colindantes a una
tarea que requiera una respuesta inminente. Busco el horario de un
tren, el vuelo de un avión, el teléfono de un hotel; guardo en un
papel los productos de cocina que he de comprar, las ideas en una
servilleta, un número de móvil en un trozo de papel, en la esquina
de un billete de metro o en la agenda de la PDA. Buscar y guardar
son actividades que la red ha fijado como una extensión natural de
nuestras tareas cotidianas. Utilizamos los buscadores web o de
escritorio, con mayor o menor acierto y éxito, para buscar o
recuperar todo tipo de información, sea nuestra o de terceros,
vista con anterioridad o no. Los motores de búsqueda no superan
nunca el umbral del segundo en proporcionar los resultados.

\begin{quote}
But as people get more sophisticated at search they are coming to
us to solve more complex problems. To stay on top of this, we have
spent a lot of time looking at how we can better understand the
wide range of information that's on the web and quickly connect
people to just the nuggets they need at that moment. We want to
help our users find more useful information, and do more useful
things with it%
\footnote{\href{http://googleblog.blogspot.com/2009/05/more-search-options-and-other-updates.html}{More Search Options and other updates from our Searchology event, 2009}}

\end{quote}
Buscar en la actual geografía del océano de información es una
actividad que no sólo responde a una única finalidad -encontrar la
información adecuada a la necesidad del momento-. En el proceso de
búsqueda interactuamos con la información; no hay búsqueda que no
venga acompañada de una cierta dosis de serendipidad o de
procrastinación. Entre encuentros fortuitos y el puro
entretenimiento por descubrir novedades, relacionadas o no con el
objeto de la búsqueda.

Buscar información es un ejercicio que William Jones (2008)
desglosa en dos grandes bloques:

\begin{itemize}
\item
  \textbf{La información vista con anterioridad}
  (\emph{¿Dónde guardé el documento?}). Recuperamos una información
  -controlada por nosotros en cuanto a que está en un espacio de
  nuestra propiedad- que guardamos en el ordenador del trabajo, en un
  dispositivo externo de memoria, en la cámara digital, en el
  portátil, o en un servicio externo controlado por terceros, es
  decir, una información que reside en alguno de los servicios web en
  los que participamos.
\item
  \textbf{La información que desconocemos su existencia}
  (\emph{¿Encontraré la imagen relacionada con el tema?}) y que no
  controlamos.saltarán
\end{itemize}
\begin{quote}
Quien necesita hoy información abundante (incluso de cierta
solvencia), teclea en Google lo que persigue y lo obtiene en
cuestión de segundos: desde entradas de blog a tesis doctorales en
línea. Lo queramos o no, en los hangares de Google está la memoria
del planeta, y la memoria es gran parte de la identidad%
\footnote{Mora, Vicente Luis. (2009)}

\end{quote}
El paisaje actual de exceso de información, de servicios de la web
y de dispositivos traza una serie de limitaciones que perjudican el
éxito (\emph{Vamos a tener suerte}) de los dos bloques arriba
mencionados, de la exploración como la recuperación de información
personal. La fragmentación, el polimorfismo, la proliferación de
diferentes tipos de información, la acumulación desorganizada, la
opacidad, la descontextualización y la duplicación de la
información son las principales barreras que la Web 2.0 intenta, de
alguna manera, resolver con mayor o menor éxito. Veamos qué
significan cada una de ellas.

\subsection{El paisaje de la fragmentación}

\begin{quote}
Our information may be scattered across various computers and
gadgets. Some information, for example, may be on a laptop computer
we use at home, other information may be on a desktop computer we
use at work and one or more PDA or smart phones. Even on a single
computer, our information is scattered across the computer desktop,
``My Document'', file folders, email folders, collections of
bookmarks, etc. New applications introduce still more forms of
organization with little or no integration to previous forms.
People can rightly complain that they have too many organizations%
\footnote{Jones et al., (2006)}

\end{quote}
La fragmentación de la información personal es consecuencia de un
paisaje en el que proliferan tanto los dispositivos físicos como
las aplicaciones que generan y guardan la información. La
información personal navega entre varios ordenadores,
\emph{smartphones}, PDA, cámaras de video, móviles, reproductores
de música, unidades de memoria externas. ¿Y la web? Gracias a la
conversación actual, la fragmentación adquiere una nueva dimensión
hasta ahora desconocida. Publicar y compartir información personal
nunca ha sido tan extremadamente fácil en la web gracias a la
transparencia técnica de estos servicios. Luego la información
personal se diluye en todas aquellos servicios en los que
continuamente generamos contenido. ¿Cómo abordar entonces la
búsqueda de información personal en este nuevo contexto? ¿Cómo
recuperar una información leída, vista con anterioridad, donde la
diversidad de dispositivos, sistemas de organización de archivos,
\emph{software} tradicional de escritorio y la \emph{nube} de la
web coexisten e interactúan cada vez con mayor frecuencia y
transparencia?

Jones (2008) considera la fragmentación de la información como un
peaje a la innovación permanente en la creación de nuevos
dispositivos, nuevas herramientas, nuevas aplicaciones web, nuevos
formatos de archivos, nuevos soportes. Pero la innovación es
consciente del problema que representa la fragmentación de la
información. Los clientes potenciales de un nuevo dispositivo, de
un nuevo servicio web, no son ajenos al significado de fragmentar y
dispersar una vez y otra su información personal. Consideremos
ahora dos escenarios habituales que provocan la fragmentación de
información:

\begin{itemize}
\item
  La \textbf{dispersión digital} de un mismo objeto de información
  repartido por los diferentes espacios de trabajo. Un ejemplo: acabo
  de preparar una presentación que ha de acompañar una sesión de
  trabajo. Esta presentación la he realizado con Microsoft
  PowerPoint, luego guardaré una copia en el formato nativo -PPT- en
  el sistema de archivos personal del portátil -primera copia-. La
  guardaré también en el disco duro externo -segunda copia-, pero
  incluiré una versión en PDF -tercera copia- ya que es posible que
  el ordenador instalado en la sala de presentaciones no disponga de
  las fuentes de letra que he utilizado. Enviaré la presentación como
  adjunto de un correo electrónico -cuarta copia- y la publicaré en
  un servicio web de presentaciones -quinta copia- del tipo
  \href{http://slideshare.ne}{SlideShare}.
\item
  A la dispersión se le ha de sumar la \textbf{proliferación} de
  versiones de un mismo documento. Su actualización o revisión
  implica una duplicación: el original y el actualizado. Crear un
  histórico de versiones del documento, siguiendo esta metodología,
  equivale a la multiplicidad del original. Si además compartimos
  este documento, la fragmentación de las versiones se acentúa
  todavía más. ¿Cual de nosotros, del grupo de personas que trabaja
  en un mismo documento, tiene la versión correcta? ¿Dónde he
  guardado la última versión correcta del documento?
\end{itemize}
Veamos cómo la innovación es sensible a esta cuestión y estudia
soluciones transparentes que hagan frente a la dispersión de la
información personal. La empresa Apple anuncia el servicio
MobileMe%
\footnote{\href{http://www.apple.com/es/mobileme/}{http://www.apple.com/es/mobileme/}}
bajo la consigna de

\begin{quote}
La forma más sencilla de mantener todo sincronizado (\ldots{}) Tal
vez tengas un Mac en casa, un PC en el trabajo y un iPhone o un
iPod touch, pero\ldots{} ¿cómo los mantienes todos sincronizados?
Con MobileMe, tu correo electrónico, contactos y calendarios
siempre son los mismos, sin importar el dispositivo que utilices.

\end{quote}
No es el único servicio de estas características. Pogoplug%
\footnote{\href{http://www.pogoplug.com/}{http://www.pogoplug.com/}}
es un dispositivo de memoria que

\begin{quote}
connects your external hard drive to the Internet so you can easily
share and access your files from anywhere. The Pogoplug is the
perfect accessory to your connected life. Imagine accessing all
your files and media from any laptop or desktop computer, anywhere
in the world

\end{quote}
A una escala menor, el laboratorio de la fundación Mozilla, trabaja
en el desarrollo de Weave%
\footnote{\href{https://services.mozilla.com/}{https://services.mozilla.com/}}

\begin{quote}
Weave is a Mozilla Labs experiment to explore opportunities for the
Web browser to broker richer experiences while increasing user
control over their data and personal information.

\end{quote}
que permitirá sincronizar el navegador Mozilla Firefox en los
diferentes dispositivos que utilizamos a diario como los
marcadores, \emph{cookies}, contraseñas, el histórico de
navegación, pestañas abiertas,\ldots{}

Los servicios como DropBox%
\footnote{\href{http://www.getdropbox.com/}{http://www.getdropbox.com/}}
o el reciente Ubuntu One%
\footnote{\href{https://ubuntuone.com/}{https://ubuntuone.com/}}
ratifican una línea de servicio idéntica: vayas a donde vayas,
utilices el dispositivo que utilices, la información estará
perfectamente sincronizada. El escritorio, multiplicado por el
efecto del peaje de la innovación y consumo tecnológico, acaba
siendo único e independiente de las herramientas que se utilicen.
La información regresa a un espacio común y, por lo tanto, se evita
la \emph{siembra} de la información por todos los dispositivos que
utilizamos a diario. La recuperación de información personal,
misión imposible para los motores de búsqueda de escritorio en un
escenario fragmentado, puede regresar con fuerza. La soluciones
como Google Desktop%
\footnote{\href{http://desktop.google.com/}{http://desktop.google.com/}},
Apple Spotlight o Windows Desktop Search%
\footnote{\href{http://www.microsoft.com/windows/products/winfamily/desktopsearch/default.mspx}{http://www.microsoft.com/windows/products/winfamily/desktopsearch/default.mspx}},
orientadas con exclusividad a satisfacer búsquedas en el escritorio
tradicional, saltarán al nuevo escritorio personal que yace en la
\emph{nube}, en consonancia con los nuevos servicios de la
sincronización entre dispositivos.

\begin{quote}
Whether you're searching your computer or the web, Desktop helps
you find it by searching your Gmail and web history along with your
hard drive. Also because your index is stored locally on your own
computer, you can even access your Gmail and web history while
you're offline.

\end{quote}
\subsection{Ayudantes}

Recuperar la información que anteriormente hemos visto es una
actividad poliédrica e insoportablemente repititiva (Kelly, 2008).
De un tiempo a esta parte, los buscadores han ido incorporando
ayudantes(\emph{helpers}) que facilitan, en la medida de lo
posible, el éxito de la búsqueda. Se adopta, en lugar de
formularios barrocos, la sintaxis de la línea de comandos; se
representa los filtros de búsqueda mediante un conjunto limitado de
operadores. Un ejemplo práctico, a veces poco conocido por el
usuario medio, es la sintaxis del buscador de Google -explicado y
reiterado por muchos, de los cuales destaco LifeHackers%
\footnote{\href{http://lifehacker.com/339474/top-10-obscure-google-search-tricks}{http://lifehacker.com/339474/top--10-obscure-google-search-tricks}}:

\begin{verbatim}
site:google.com filetype:pdf web2.0
\end{verbatim}
Otro ejemplo, no menos ilustrativo, es la sintaxis de búsqueda del
servicio de correo de Google -GMail-, donde encontramos
expresiones%
\footnote{\href{http://mail.google.com/support/bin/answer.py?hl=en&answer=7190}{http://mail.google.com/support/bin/answer.py?hl=en\&answer=7190}}
del tipo:

\begin{verbatim}
from:mai in:anywhere has:attachment filename:pdf
after:2007/04/16 
\end{verbatim}
que vendría a decir
\emph{busca todos los mensajes enviados por correo que adjunten un archivo del tipo pdf después de la fecha del 16 de abril de 2007}

Veamos otro ejemplo. El buscador de Twitter nos proporciona una
lista de operadores%
\footnote{\href{http://search.twitter.com/operators}{http://search.twitter.com/operators}}
que nos ayudará a recuperar \emph{tweets} pasados (pero no más allá
de los límites que impone el propio servicio).

\begin{verbatim}
flight :( containing "flight" and with
a negative attitude
\end{verbatim}
La acción de autocompletar la expresión que estamos escribiendo en
el formulario de búsqueda es uno de los ayudantes más
generalizados. Nos corrige posibles errores de ortografía y, a un
mismo tiempo, nos sugiere un resultado idóneo o colateral al objeto
de la búsqueda. Las variaciones del autocompletado son múltiples:
la palabra clave, la cuenta de correo electrónico, la página web, e
incluso las contraseñas.

Pero estos dos ejemplos de ayudantes tienen una serie de
limitaciones:

\begin{itemize}
\item
  El sistema de autocompletado no sabe de errores humanos. No
  diferenciará si una búsqueda está bien o mal escrita, luego, si
  escribimos una búsqueda que previamente se ha escrito
  equivocádamente, el sistema lo recuperará, lo que proporcionará -si
  es recurrente- una desagradable experiencia al usuario.
\item
  El autocompletado es \emph{dependiente del dispositivo}, luego no
  sortea el problema de la fragmentación de información personal.
\item
  Los buscadores no diferencian cuál es la versión correcta del
  histórico de un documento.
\item
  Los buscadores no pueden indexar contenidos audiovisuales, luego se
  apoyarán en los metadatos asociados a estos contenidos.
\end{itemize}
\begin{quote}
``You press the button , We do the rest'' Kodak advertisement

\end{quote}
En la actualidad no utilizamos un único navegador: el ordenador de
sobremesa, el portátil personal, el del trabajo, el dispositivo
móvil, el ordenador de un aula de formación, el de la biblioteca.
Deseamos que los dispositivos resulten transparentes cuando
buscamos y accedemos a nuestro espacio personal de información.
Pero ¿dónde situar la frontera que separa la información personal
de la colectiva? En la conversación, el límite entre la esfera
privada y la colectiva se difumina hasta desaparecer. Si
participamos activamente en la redacción y corrección de artículos
en la Wikipedia, ¿no es un tipo de información personal -el
conocimiento sobre un tema- que incorporamos a una estructura
mayor, colectiva? Gestionar los marcadores en un servicio como
Delicious, publicar artículos y comentarios en la
\textbf{blogosfera}, subir un video a Vimeo o YouTube, ¿no rompemos
definitivamente con los estrechos márgenes del escritorio personal
y delegamos a los nuevos servicios web, marca 2.0, la gestión de
nuestra información? La seducción de la facilidad de uso -no
necesito comprar ningún manual o recibir formación previa- y el
acceso desde cualquier dispositivo conectado -evito la mochila
digital- nos convence definitivamente. Mi entrada a este singular
escritorio colectivo sólo requiere de un navegador. Es evidente que
el mercado se dirige a este escenario. De ahí, el éxito de ventas
de los portátiles ligeros, -menos prestaciones técnicas, pero las
suficientes que permitan un acceso a mi escritorio colectivo-,
móviles que dejan de serlo a favor de los \emph{smartphones}, o la
adecuación o innovación de sistemas operativos como Moblin%
\footnote{\href{http://moblin.org/}{http://moblin.org/}},
Ubuntu MID%
\footnote{\href{http://www.ubuntu.com/products/mobile}{http://www.ubuntu.com/products/mobile}}
o Android%
\footnote{\href{http://www.android.com/}{http://www.android.com/}}.

\begin{quote}
While everyone was searching, there was bailing," a narrator says.
"While everyone was lost in the links, there was collapsing. We
don't need queries and keywords if they bring back questions and
confusion. From this moment on, search overload is officially over%
\footnote{Microsoft's Bing Commercial:
\href{http://www.webguild.org/2009/06/microsofts-bing-ad-claims-to-terminate-search-overload.php}{http://www.webguild.org/2009/06/microsofts-bing-ad-claims-to-terminate-search-overload.php}}

\end{quote}
\subsection{Buscar en la conversación}

Pero en términos de búsqueda, ¿qué buscar en la conversación? De
igual manera que no registramos a diario nuestras conversaciones
cotidianas, ¿es necesario recuperar nuestra voz de hace un mes, un
año? La búsqueda se plantea no en clave de pasado sino en
presente.

\begin{quote}
See what's happening - right now%
\footnote{Eslogan de Twitter Search}

\end{quote}
La búsqueda de la \emph{Real Time Web} es la búsqueda en la
conversación. ¿Qué significa? No se refiere a indexar documentos,
páginas, archivos, imágenes o fotografías de la web. El servicio de
microblogging Twitter ha demostrado que su buscador no tiene nada
que ver con la recuperación de viejos \emph{tweets} de los
usuarios. El buscador de Twitter%
\footnote{\href{http://search.twitter.com/}{http://search.twitter.com/}}
actúa de sismógrafo de la conversación, traza sobre la pantalla las
tendencias que estan ocurriendo en ese momento en la conversación.
Twitter detecta a tiempo real cualquier movimiento sísmico que
pueda provocar la conversación.

\begin{quote}
I have always thought we needed to index the web every second to
allow real time search. At first, my team laughed and did not
believe me. Now they know they have to do it. Not everybody needs
sub-second indexing but people are getting pretty excited about
realtime%
\footnote{\emph{Larry Page about Twitter}
(2009)\href{http://www.loiclemeur.com/english/2009/05/larry-page-about-twitter.html}{http://www.loiclemeur.com/english/2009/05/larry-page-about-twitter.html}}

\end{quote}
No resulta difícil imaginar la expresión de los desarrolladores de
Google al conocer la propuesta de Larry Page. Aparte del buscador
de Twitter, ejemplos como el de OneRiot%
\footnote{\href{http://www.oneriot.com}{http://www.oneriot.com}},
Scoopler%
\footnote{\href{http://www.scoopler.com/}{http://www.scoopler.com/}},
TweetMeme%
\footnote{\href{http://tweetmeme.com/search.php}{http://tweetmeme.com/search.php}}
o el buscador de FriendFeed%
\footnote{\href{http://friendfeed.com/search}{http://friendfeed.com/search}},
inciden en el hecho que la mayoría de los usuarios buscan el pulso
de la conversación, toda información de la que la gente habla,
comparte o está simplemente consultando. El \emph{PageRank} deja
paso a una suerte de \emph{SocialRank}:

\begin{quote}
OneRiot crawls the links people share on Twitter, Digg and other
social sharing services, then indexes the content on those pages in
seconds. The end result is a search experience that allows users to
find the freshest, most socially-relevant content from across the
realtime web.

\end{quote}
Pero los buscadores de la conversación no tienen memoria. Twitter
limita las búsquedas a un mes vista del día actual. Guardar la
conversación más allá de un límite de tiempo no es relevante,
incluso si es la conversación de uno mismo. Experimentos como
Archivist%
\footnote{\href{http://www.flotzam.com/archivist/}{http://www.flotzam.com/archivist/}}
tienen un acento que roza lo antropológico. Buscar en la
conversación de la web persigue la instantaniedad del momento, no
su fijación o preservación.

\begin{quote}
The Web is alive. Why are we still searching like it is 1999?%
\footnote{Anuncio de Collecta
\href{http://collecta.com/}{http://collecta.com/}}

\end{quote}
\subsection{Preservar}

\begin{quote}
I think of those old Apple II disks in my closet. My son's first
Basic program, still on a Commodore 64 cassette. I flip through my
filing cabinet. Sure enough, some 51/4-inch floppies for an IBM PC,
containing who knows what? I think of a stele in the Louvre, with
Hammurabi's code of law carved on it 3,000 years ago, of listening
to the English translation on the digital recorder you rent at the
museum door. I think of Shakespeare's first folio, fading but
readable%
\footnote{Gulie, Steven
\emph{Saved. What death can't destroy and how to digitize it}
\href{http://wired-vig.wired.com/wired/archive/6.09/saved.html?pg=1&topic=}{http://wired-vig.wired.com/wired/archive/6.09/saved.html?pg=1\&topic=}}

\end{quote}
En la primera época de la informática doméstica, las cintas de
cassette fueron los principales dispositivos de memoria. Un
conversor del ruido análogico -la cinta en funcionamiento en un
reproductor de cintas de casete- a digital traducía las señales de
sonido al lenguaje de la máquina. El ruido chirriante, altisonante,
pasados unos minutos, se transformaba en información, en un texto,
en un gráfico, o en un juego en el televisor. Steven Gulie compara
las cintas de casete, los primeros discos, con el silencio de los
restos de civilizaciones desaparecidas que perduran en los museos
de arqueología. Los programas escritos en el lenguaje BASIC son
ruido si sólo los reproducimos en un radiocassette de la época, o
silencio si ni siquiera disponemos del reproductor adecuado. Como
un grabado de alquimia del siglo XVII, el verdadero significado
permanece oculto; la información sólo es recuperable a través de
técnicas actuales de emulación. Pura arqueología digital.

\begin{quote}
Pero la tendencia actual a pasar todo tipo de contenidos a formato
digital tampoco está exenta de problemas, porque aparte de poder
acceder físicamente a los datos -aún tengo por casa cintas de
casete con programas Commodore 64 y disquetes de 5 1/4 que hace
años que no tengo forma de leer-, el problema es que muchos de los
contenidos guardados están en formatos obsoletos creados con
programas que hoy en día ya no existen, con lo que aunque los datos
están ahí y probablemente aún se puedan leer, no hay forma de
interpretarlos%
\footnote{Microsiervos (2009) \emph{El cuento de nunca acabar}
\href{http://www.microsiervos.com/archivo/ordenadores/vdg-el-cuento-de-nunca-acabar.html}{http://bit.ly/Aluk0}}

\end{quote}
El secuestrador austríaco que retuvo a Natascha Kampusch parte de
su vida, Wolfgang Priklopil, sorprendió a la policía cuando ésta
descubrió un viejo Commodore 64. Se consideró la posibilidad que
pudiera guardar información relevante para comprender la actitud
criminal del secuestrador, pero la recuperación de la información
no sería trivial debido a que no estaría exenta de pérdida e
integridad de los datos, según informó la Policía austríaca a los
medios de comunicación. Slashdot%
\footnote{\href{http://slashdot.org/articles/06/09/05/2028258.shtml}{http://slashdot.org/articles/06/09/05/2028258.shtml}}
se hizo eco de la noticia; entre la interminable lista de
comentarios asociados a la macabra noticia, un usuario se
preguntaba

\begin{quote}
how all of this data that isn't transferred to modern day systems
will be dealt with, both from a technological and machine- and
media-aging point-of-view. And in the context of criminal
investigations, what happens if evidence is ``lost'' (or simply
unrecovered) from a 25-year-old computer in a murder investigation
which has no statute of limitations? It's an equally difficult
question for governments, corporations, and academic institutions
that actually \emph{want} to keep the data but are having trouble
instituting standards, policies, and mechanisms for data
retention.

\end{quote}
A una escala mayor y en un sentido completamente diferente, el
proyecto \emph{BBC Domesday} representó, en la década de 1980, una
reinterpretación del trabajo, 900 años después, que impulsó el rey
Guillermo I en la creación del primer censo conocido de las
propiedades de la población de Inglaterra. Si aquél se concretó en
dos volúmenes, el \emph{little Domesday} y el
\emph{great Domesday}, el homenaje que propuso la BBC, en el que
participaron alrededor de un millón de escolares. Los estudiantes
contribuyeron al proyecto escribiendo sobre la historia, geografía
y acontecimientos de importancia desde una perspectiva local. La
información, repartida en mapas, videos, fotografías, interactivos
y escritos, quedó literalmente atrapada en dos \emph{laserdiscs} de
la época. Poco tiempo después, el proceso de obsolescencia del
dispositivo de almacenamiento y el tipo de formato en el que se
registró la información, provocó la alarma, sabedores de tener
frente a ellos una información irrecuperable. Se originó una serie
de iniciativas de extracción y migración de la información a las
tecnologías y formatos actuales, no sin enormes dificultades. El
error y, por lo tanto, lectura, en palabras de Mike Tibbets,
responsables del proyecto fue

\begin{quote}
not in the lack of vision or foresight by the technologists but
that, at least in the UK, the national systems of data preservation
and heritage archiving simply don't work reliably or consistently%
\footnote{\href{http://bit.ly/ctg0h}{http://bit.ly/ctg0h}}

\end{quote}
Tendemos a preservar el exceso de información, aunque la mayoría de
ella sea inútil o irrelevante. Toda información está bajo sospecha
de ser valiosa en un futuro, próximo o lejano, luego no se destruye
nada. Se pretende que toda información, por estúpido o banal que
resulte su contenido, sea siempre recuperable en un futuro.

\begin{quote}
If so much of our personal history is getting compressed into data,
and digital imaging, cloud computing, and streaming media have
become an integral part of daily experience, being sensitive to the
physical presence of these devices is an important responsibility%
\footnote{Diana, Carla (2009) \emph{Atoms For Bits}
\href{http://www.core77.com/blog/featured_items/atoms_for_bits_designing_physical_embodiments_for_virtual_content_13625.asp}{http://www.core77.com/blog/featured\_items/atoms\_for\_bits\_designing\_physical\_embodiments\_for\_virtual\_content\_13625.asp}}

\end{quote}
Jones (2008) considera que la actividad de preservar la información
responde a más de una cuestión: ¿dónde guardarla? ¿en qué formato?
¿en qué dispositivo? Los servicios web actuales aligeran la
necesidad de tomar una decisión al respecto, ya que si bien antes
podría haber claras limitaciones físicas de espacio (comprar libros
resulta un problema) o cuando los dispositivos de almacenamiento
eran limitados, ahora resulta una práctica de cazadores de
información, incluso aunque la información en sí misma no resulte
útil. No guardamos copia de esta información, sólo instancias de la
misma: de las páginas web guardamos su marcador, de aquellas
fotografías, presentaciones o videos los marcamos de nuestro
interés; marcamos/cazamos comentarios, grupos de discusión,
noticias, opiniones, enlaces, vuelos, rutas, hoteles,
estancias,\ldots{}; agregar noticias no deja de ser también una
actividad relacionada con guardar la información. No duplicamos,
realizamos instancias de la infomación. Toda información de nuestro
interés ha dejado de formar parte de nuestros límites físicos. Ya
no es de nuestra propiedad. Guardarla, descargarla -y así parece la
máxima instrucción de la marca 2.0- no tiene sentido. Genera
información. Publica, escribe, valora, sugiere, sube tu colección
de fotografías, de videos. Súbelo todo a la \emph{nube}.
\emph{Nosotros nos encargamos del resto}.

\chap{Folders \& Tagging redundancy}

\begin{quote}
I don't use the word folksonomy. Tagging in delicious is about 1/3
classification and 2/3 functionality. Something easy to do that
let's you recall the item. The goal isn't to classify, it's to
remember%
\footnote{Blair, Ann. (2003).
\emph{Coping with Information Overload in Early Modern Europe}}

\end{quote}
Organizamos la información con el propósito de recuperarla más
adelante; la clasificamos en la medida que nos ayude a comprenderla
una vez recuperada. Preparar un proyecto que aglutine documentos de
texto, hojas de cálculo, fotografías, videos, presentaciones
significa ordenar a un tiempo cad uno de los elementos que lo
forman siguiendo una estructura previa que otorgue un sentido a
cada una de las partes. Esto mismo, en términos digitales, lo
traducimos en un sistema de carpetas y subcarpetas anidadas.

En un escenario ideal, seguir pautas claras de organización nos
evitaría momentos de enorme confusión: ¿dónde he guardado el
archivo que necesito? Pero las circunstancias son otras y la lógica
misma de un sistema de clasificación personal acaba, con el tiempo,
degradándose y perdiendo toda su efectividad. Existe el problema de
flexibilidad y adaptación de un sistema rígido de clasificación de
información en un entorno cambiante, excesivamente cambiante como
es el espacio personal de información de nuestros dispositivos de
uso cotidiano. Dejamos de seguir la convención con la que ordenamos
la información: el nombre, identificador o numeración de carpetas y
subcarpetas; vamos dejando los múltiples archivos en el escritorio,
más tarde los desplazamos a lugares que no siguen ninguna lógica
organizativa, no forzamos a que las aplicaciones guarden un adjunto
o documento en el directorio que uno quiere, sino en aquel que
tiene por defecto. Esta degradación aumenta si pensamos que la
tipología de archivo mantiene la metáfora del documento de papel.
Un ejemplo: la aplicación de búsqueda en local, Google Desktop
Search%
\footnote{Wilson, (2005).
\emph{\href{http://www.newscientist.com/article/mg18624973.400}{Info-overload harms concentration more than marijuana}}}
asegura indexar todo tipo de elementos: correo electrónico,
documentos de texto, hojas de cálculo, presentaciones, las
conversaciones de la mensajería instantánea, documentos
encapsulados (PDF), archivos comprimidos, así hasta rozar
trescientos formatos diferentes de archivos. Una realidad que se
sitúa en las antípodas del escritorio tradicional.

\begin{quote}
In traditional file hierarchies, the user can only provide simple
annotations. For example, people must name their folders and files.
They may optionally specify additional information, such as a short
text comment or a highlight color, which are stored as file system
metadata. Other annotations are automatically supplied by the
system, including the icon associated with the file type, the date
the file was last modified, and, perhaps, a thumbnail view of the
file's contents. The problem is that these annotation capabilities
are extremely limited, given the volume and diversity of files that
most individuals must manage. (\ldots{}) So, while file browsers
are excellent for allowing people to create, modify, move, and
delete files, they provide extremely limited tools for annotating
these artifacts%
\footnote{Hurst, Mark. (2007). \emph{Bit Literacy}. Good Experience Press.}

\end{quote}
Kebin Kubasik, desarrollador del indexador libre Beagle%
\footnote{Hallowell, E. M. (2005).
\emph{\href{http://tr.im/wSsS}{Overloaded circuits: Why smart people underperform}}}
en la entrada
\emph{The Reality of Semantic Desktops: Death To Tags, Labels and Folders}%
\footnote{Malhotra, Naresh K. (1984).
\emph{\href{http://www.jstor.org/pss/2488913}{Reflections on the Information Overload Paradigm in Consumer Decision Making}}}
defiende un nuevo escritorio que, definitivamente, ponga punto y
final a la metáfora del escritorio tradicional, que rompa la
sinergia de continuar la tradición de establecer jerarquías de
directorios, categorías, etiquetas, distintivos de color y
cualquier otro emblema visual con fines organizativos. Toda
práctica de clasificación de la información personal y profesional
siguiendo este modelo termina en fracaso

\begin{quote}
how many people actually keep their address book completely
updated? Or tag all their photos, or keep every document in the
right folder? Even those who are vigilant eventually fall behind,
and that's because users already know what the material they are
filling is, but still have to spend time explaining to the computer
which items are related and where they belong.

\end{quote}
No son pocas las voces que denuncian las limitaciones en términos
de \emph{findability}%
\footnote{Jacoby, J. (1984),
\emph{\href{http://www.jstor.org/pss/2488912}{Perspectives on Information Overload}}}
del actual sistema de carpetas y archivos. No nos referimos a la
información que \emph{no está bajo nuestro control}, sino de
aquélla que guardamos tanto en dispositivos personales
(ordenadores, PDA, memorias externas, Smartphones,\ldots{}) como
también aquélla que, si bien la consideramos nuestra, no está
completamente bajo nuestro control, dependiendo de terceros (por
ejemplo, los servicios de la \emph{nube}) En relación a esta
cuestión, Paul Sherman%
\footnote{Schneider, Ursula. (2002).
\emph{\href{http://www.jucs.org/jucs_8_5/the_knowledge_attention_gap/Schneider_U.html}{The Knowledge-Attention-Gap: Do We Underestimate The Problem Of Information Overload In Knowledge Management?}}},
diferencia tres puntos de tensión:

\begin{itemize}
\item
  ¿Cómo organizar la información que guardo para que pueda
  recuperarla más tarde sin dificultad?
\item
  ¿Puedo encontrar con facilidad aquella información que he guardado
  con anterioridad?
\item
  ¿Puedo modificar fácilmente la organización de mi información?
\end{itemize}
Las tres preguntas han originado todo tipo de respuestas: desde los
ya habituales consejos de auto-ayuda digital%
\footnote{Barthes, Roland. 1980. \emph{La cámara lúcida}. Paidós}
a visiones de escritorios 3D, de los cuales remarcar Bumptop%
\footnote{Italo Calvino, \emph{Las ciudades invisibles}},
más por su efectismo que por su efectividad. Pero como ya hemos
señalado más arriba, cualquier ordenación personal basada en
carpetas y subcarpetas acaba tarde o temprano en un proceso de
degradación. Prueba de ello la encontramos en la expresión más
superlativa de la desorganización de la información: los
\emph{clutter desktop} escritorio inundado de archivos y carpetas
que apenas dejan ver si quiera la imagen de fondo utilizada para el
escritorio. Este final de camino de la desorganización queda
artísticamente representado en concursos informales%
\footnote{\href{http://en.wikipedia.org/wiki/Personal_information_management}{http://en.wikipedia.org/wiki/Personal\_information\_management}}.
No es extraño que se potencie y promocione el uso de buscadores de
escritorio que reemplacen la falta de una organización personal de
información (y resuelvan, por lo tanto, el problema de recuperar
independientemente del dispositivo en que se encuentre). Pero, ¿y
la información personal que flota en la \emph{nube}? La actividad
en la red, cada vez más intensa, hace que resulte más complicada la
convivencia entre una organización tradicional de documentos y
carpetas de escritorio y la actividad social en la conversación. El
servicio Plaxo%
\footnote{minid.net (2009). \emph{En casa de herrero, cuchillo de palo}
\href{http://www.minid.net/2009/06/17/en-casa-de-herrero-cuchillo-de-palo/}{http://www.minid.net/2009/06/17/en-casa-de-herrero-cuchillo-de-palo/}}
es un ejemplo de ello. Agrupa información tradicional del tipo
tareas, calendario, agenda, fotografías, videos, pero también
incluye la opción de sumar nuestra voz de la conversación agregando
aquellos servicios web que utilizamos. En cierta medida, se vuelve
invisible el organizador -las carpetas, los archivos, los formatos
de los archivos- y se hace más tangible la información. Ya no más
archivos o carpetas. Como mucho, ¿etiquetas?

\section{Organizar la conversación}

La aparición a primeros de 2003 del etiquetaje o \emph{tagging}
como estrategia de clasificación personal y colectiva de
información vino de la mano del servicio de marcadores Delicious.
Falta muy poco para que se bautice la extraordinaria actividad de
\emph{escribir} la web por parte de sus usuarios como
\emph{Web 2.0}. El sistema de etiquetas, las \emph{folksonomías}
(etiquetajes colectivos) es adoptado rápidamente por otros
servicios como Flickr (junio de 2004), Last.fm (Agosto de 2005),
Technorati (Enero de 2005)%
\footnote{\href{http://googleblog.blogspot.com/2009/05/more-search-options-and-other-updates.html}{More Search Options and other updates from our Searchology event, 2009}},
Youtube (Jun de 2005)%
\footnote{Mora, Vicente Luis. (2009)};
clasificar por etiquetas las entradas de un blog personal se
convierte en hábito; emergen las nubes de etiquetas
(\emph{cloud tags}) como la mejor estrategia de visualizarlas y
nadie que se preciaba estar en la vorágine del torbellino de la
marca 2.0 no descansó en evangelizar este sistema como panacea a
todos los males de la ordenación de información, personal o
colectiva.

Gene Smith%
\footnote{Jones et al., (2006)}
describe, en términos generales, los beneficios de la práctica del
\emph{tagging}:

\begin{itemize}
\item
  \textbf{Facilidad de uso}: La simplicidad y flexibilidad que nos
  permite organizar la información frente a una ordenación jerárquica
  en carpetas y subcarpetas.
\item
  \textbf{PIM}: La gestión de información personal tiene un recorrido
  inverso: de local -etiquetar primero- a general -visualizar la
  estructura general en base al conjunto de etiquetas-. Esta
  diferencia es visible si recuperamos la comparación entre ordenar
  con etiquetas o carpetas o, en palabras de Alex Wright%
  \footnote{\href{http://www.apple.com/es/mobileme/}{http://www.apple.com/es/mobileme/}},
  la tensión permanente entre dos tipologías de organización cuasi
  ancestrales: la jerárquica -disposición vertical-, en forma de
  árbol invertido, y la horizontal, en forma de red. Si la primera
  tipología es sinónimo de rigidez y fácilmente reconocible en la
  estructura de directorios o carpetas del sistema operativo o de
  algunos clientes de correo, la segunda tipología correspondería a
  la práctica del \emph{tagging}. De nuevo se destaca la flexibilidad
  con la que nos permite estructurar la información y, sobretodo,
  categorizarla a nuestra medida. Otro ejemplo de valor añadido es no
  duplicar la información si esta pertenece a dos categorías
  diferentes. Simplificando: si seguimos un sistema rígido basado en
  carpetas y subcarpetas, es posible toparnos con una situación en la
  que un documento, por su siginicado o temática, deba pertenecer a
  dos o tres carpetas diferentes, lo que acaba por degradar el propio
  sistema de clasificación al provocar que cierta documentación
  necesite duplicarse o triplicarse. En su contra, gracias al
  etiquetaje, asociamos al documento las etiquetas que sean
  necesarias sin necesidad de recurrir a su copia.
\item
  \textbf{Colaborar, compartir}: Ejemplos como Flickr o Delicious en
  el que las etiquetas forman parte de la idiosincrasia del servicio,
  ayudan a los usuarios a descubrir o explorar nuevos contenidos a
  través de sus etiquetas.
\item
  \textbf{Expresión personal}: Etiquetar es, de alguna manera,
  definir de manera subjetiva un contenido. La actividad de etiquetar
  no es imparcial ni es resultado de un plan organizativo previo. En
  esta dirección, el \emph{tagging} puede originar movimientos de
  denuncia, como por ejemplo el emprendido por la organización
  defectivebydesign.org%
  \footnote{\href{http://www.pogoplug.com/}{http://www.pogoplug.com/}}
  contra el DRM%
  \footnote{\href{https://services.mozilla.com/}{https://services.mozilla.com/}}
  al utilizar el servicio de \emph{tagging} de Amazon para
  identificar todo aquel producto que estuviera sujeto bajo DRM.
\end{itemize}
Al margen de estos usos benéficos de las etiquetas, otras
opiniones, contrarias al \emph{tagging} han ido manifestando sus
limitaciones a diferentes niveles.

\begin{quote}
Tagging in 2007 seems to have advanced no further than a means by
which one or more users of a site (or application) can group
content around a loose framework of concepts. Matt Mower%
\footnote{\href{http://www.getdropbox.com/}{http://www.getdropbox.com/}}

\end{quote}
Hurst, por el contrario, es de la opinión de no utilizar etiquetas
como método de clasificación de la información personal. Considera
que la práctica de, por ejemplo, etiquetar fotografías requiere de
un esfuerzo que, la mayoría de usuarios, no está dispuesto a
asumir. En general, muy pocos son los usuarios que sigan a
conciencia un sistema rígido de clasificación y organización de los
archivos. Un ejemplo de esto es cómo los propios usuarios descuidan
modificar el nombre que, por defecto, las bautiza la propia cámara
digital. Luego tampoco parece factible que los usuarios dediquen un
tiempo, aunque corto, a clasificarlas por etiquetas.

\begin{quote}
a million and one more ways for me to catalog and store my data,
but when I'm actually looking for something the tags never seem to
help much

\end{quote}
La denuncia de las limitaciones del uso de etiquetas como sistema
de recuperación de la información es un recurrente en la Web 2.0.
Enrique Dans%
\footnote{\href{https://ubuntuone.com/}{https://ubuntuone.com/}}
expuso con claridad la situación de
\emph{falsos-positivos y falsos-negativos}%
\footnote{\href{http://desktop.google.com/}{http://desktop.google.com/}}
y abogar por un sistema mixto de organización: convivencia entre la
vieja guardia de las categorías y el sistema flexible de etiquetaje
social. La contundente afirmación de Jesús Encinar, CEO
(\emph{Chief Executive Officer}) de Idealista.com%
\footnote{\href{http://www.microsoft.com/windows/products/winfamily/desktopsearch/default.mspx}{http://www.microsoft.com/windows/products/winfamily/desktopsearch/default.mspx}}
al hablar del rediseño del servicio 11870.com%
\footnote{\href{http://lifehacker.com/339474/top-10-obscure-google-search-tricks}{http://lifehacker.com/339474/top--10-obscure-google-search-tricks}}

\begin{quote}
Con el tiempo hemos visto que 2.500 años de historia de las
bibliotecas no están equivocadas.

\end{quote}
Se castiga así a la \emph{folksonomía} por la deriva de
\emph{inconsistencias, sinonimias, homonimias, polisemias o, en general, la falta de homogeneidad y de coherencia}
(Dans, 2008). Esta apariencia caótica de clasificación de la
información espanta a puristas de las ontologías y a quienes ven el
sistema colectivo de etiquetaje como un camino sin retorno al pero
de los escenarios posibles: la ausencia de organización. Antes de
alcanzar este fatídico escenario lleno de inconsistencias, no hay
pocas voces de ayuda en explicar buenas prácticas del
\emph{tagging}%
\footnote{\href{http://mail.google.com/support/bin/answer.py?hl=en&answer=7190}{http://mail.google.com/support/bin/answer.py?hl=en\&answer=7190}}
en las que los consejos pasan por utilizar siempre los nombres en
singular, escribir en minúscula y, en especial, nombre cortos,
simples y fáciles de recordar.

\begin{quote}
But if you look closer, you can see massive redundancy and
disorganization. There are tags for brandenthusiasts, branding and
brands. Separate ones for forum and forums. And a run of social
media related ones that include no fewer than sellingsocialmedia,
socialbookmarking, socialmedia, socialmediamarketing,
socialmediameasurement, socialmediarelease, socialmediaroi,
socialmediastrategy, socialmediatools, socialnetworking and
socialnews. Oh, and there are separate entries for business, CEO
and corporate%
\footnote{\href{http://search.twitter.com/operators}{http://search.twitter.com/operators}}

\end{quote}
Ejemplo de lo anterior: el usuario \verb!earth2marsh! es poseedor
de 5681 marcadores en Delicious; ha utilizado, por el momento, 4563
etiquetas%
\footnote{\href{http://moblin.org/}{http://moblin.org/}}
diferentes para organizar los marcadores.

Si el futuro del \emph{tagging} está bajo una cierta sospecha, Gene
Smith defiende en el artículo \emph{Tagging: Emerging Trends}%
\footnote{\href{http://www.ubuntu.com/products/mobile}{http://www.ubuntu.com/products/mobile}}
una evolución viable en base a cuatro tendencias del
\emph{tagging}:

\begin{itemize}
\item
  \textbf{Mayor control de los vocabularios}, luego incorporar una
  cierta estructura que no sacrifique o reduzca la libertad y
  flexibilidad demostrada por las \emph{folksonomías}. Insistir en
  los beneficios del \emph{tagging} frente a las taxonomías o
  vocabularios cerrados, restringidos, inflexibles y de poco valor
  social%
  \footnote{\href{http://www.android.com/}{http://www.android.com/}}.
  Cory Doctorow, en su conocido artículo
  \emph{Metacrap: Putting the torch to seven straw-men of the meta-utopia}%
  \footnote{Eslogan de Twitter Search}
  ya planteó la imposibilidad de una taxonomía global; en un
  escenario ideal, los metadatos proporcionarían la ayuda exacta para
  encontrar la información que estamos buscando, debido a que ésta
  estaría correctamente identificada; es el lo que Cory Doctorow
  identifica con la \emph{meta-utopia}, un estadio de la información
  en la que los metadatos son de una gran ayuda, imparciales y
  neutrales, pero ineficaces cuando se trata de un mundo en el que la
  gente \emph{miente}, es \emph{idiota} y \emph{vaga}. No olvidemos
  que las palabras de Cory Doctorow se escribieron en el 2001. De
  hecho, anticipa una realidad donde los metadatos se pervierten
  originando una marea tóxica de información en la red.
\item
  \textbf{Folksonomías semiautomáticas} o, lo que sería una
  invitación a los usuarios a aceptar unas etiquetas prefedefinidas
  que son, en realidad, categorías temáticas que permiten una
  ordenación más o menos estable de la información. Rashmi Sinha en
  \emph{Findability with tags: Facets, clusters, and pivot browsing}%
  \footnote{\href{http://search.twitter.com/}{http://search.twitter.com/}}
  las describió como bloques o agrupaciones -\emph{facets}- de
  etiquetas que tengan y compartan un significado más amplio o común
  a todas ellas. Encontramos ejemplos de ello en el momento de
  publicar una noticia en el servicio Menéame, el usuario debe elegir
  la categoría.
\item
  \textbf{Complicidad} con una parte de la comunidad de usuarios en
  la que ayuden a reducir el ruido producido por las inconsistencias
  derivadadas de la \emph{folksonomía}.
\item
  El lado \textbf{creativo} e innovador de la práctica del
  \emph{tagging}. Ya hemos mencionado más arriba el ejemplo de
  denuncia contra el DRM por parte de la organización
  \textbf{defectivebydesign.org}.
\end{itemize}
En cierta manera, si las etiquetas han acaparado buena parte de la
clasificación de los servicios de la conversación, no es así en el
caso del escritorio personal, donde la propia organización del
sistema operativo en carpetas y archivos -si bien es técnicamente
posible-, no nos resulta un escenario propicio para etiquetar
\emph{nuestras} fotografías, documentos, videos, canciones,\ldots{}
El valor de las etiquetas es probablemente social y no como
elemento organizador de la información.

\section{Entre la voz y la conversación: el relato}

\begin{quote}
Suppose there were a way to combine storytelling and organizing? No
matter how many stories we tell, our offices, basements and closets
will be just as messy as before. But the story, as it were, for our
digital information is different%
\footnote{\emph{Larry Page about Twitter}
(2009)\href{http://www.loiclemeur.com/english/2009/05/larry-page-about-twitter.html}{http://www.loiclemeur.com/english/2009/05/larry-page-about-twitter.html}}

\end{quote}
Llegado a este punto en el que los diferentes sistemas de
ordenación de la información personal se ven superados por la
diversificación de dispositivos, fragmentación y la conversación de
la web, no resulta difícil imaginar el escenario actual como un
ecosistema en el que, no sin tensiones, conviven con resultados
irregulares las dos metodologías arriba explicadas, más otras
propias de aplicaciones o servicios web que centralizan y cautivan
la información que manejan.

Frente a esta colorida iconografía de carpetas, archivos,
etiquetas, estrellas, banderas, colores, marcas\ldots{} Jones%
\footnote{\href{http://www.oneriot.com}{http://www.oneriot.com}}
propone la técnica del \emph{Storytelling}%
\footnote{\href{http://www.scoopler.com/}{http://www.scoopler.com/}}
o \emph{arte de contar historias}, la construcción de relatos
asociados a los diferentes ítems (fotografías, videos, opiniones,
documentos de texto, marcadores,\ldots{}) como metodología de
organización de información personal. El relato proporciona un
valor añadido en cuanto a significación de aquellos objetos de
información que forman parte de él; ofrece a su vez un contexto que
facilita su comprensión (\emph{¿por qué guardé esta información?})
y su recuperación (\emph{¿dondé guardé esta información?}) ya que,
en gran medida, el relato es en sí mismo un instrumento preciso de
comunicación, persistente y persuasivo, en el pueden sumarse otras
voces en la propia construcción del relato. Lugares ubicados entre
el espacio personal y la conversación como, por ejemplo, los blogs
o wikis personales, funcionarían como generadores de relatos que,
en realidad, actuarían de organizadores de la información personal.
La clasificación de los elementos de información sería múltiple:
desde la propia clasificación por categorías, etiquetas, fechas de
publicación-actualización-modificación del mismo relato, pasando
por la indexación de los contenidos e ítems que la misma
herramienta (blog, wiki) proporciona, o de los buscadores generales
en caso de que estos relatos sean públicos y visibles.

\begin{quote}
Sometimes Newton's Law of Inertia is just as applicable to the
digital world as to the physical. All too often our files sit
around never to be filed. Downloads and other sundry files pile up
never to leave. Fortunately, an uncluttered desktop can be a
reality%
\footnote{\href{http://tweetmeme.com/search.php}{http://tweetmeme.com/search.php}}

\end{quote}
\chap{RSSocial fatigue}

\begin{quote}
I'm a devoted RSS user. I have RSS subscriptions to a couple of
hundred different blogs, and I can go through them very quickly
using my RSS reader program. Instead of having to visit all those
sites individually, I can browse a list of headlines from all my
subscribed feeds in one window%
\footnote{Blair, Ann. (2003).
\emph{Coping with Information Overload in Early Modern Europe}}

\end{quote}
Antes que el \emph{blogging} se democratizara, quien mantenía su
propia página web informaba a los visitantes de las novedades o
actualizaciones mediante algún tipo de aviso textual o visual. Los
usuarios guardaban en la carpeta \emph{Favoritos} del navegador los
marcadores de las páginas de interés con el fin de, más adelante,
visitar de nuevo la web y comprobar si uno de estos avisos visuales
animados estaba presente, informándonos que alguna parte del
contenido era nuevo. El problema surgió cuando el número de
marcadores crecía en la misma medida que nuestra curiosidad. El
tiempo que dedicábamos a averiguar si habían novedades se hacía
cada vez mayor. Esta manera de saber si las páginas habían
actualizado o no sus contenidos no resultaba nada eficiente.
Mantenerse al día de la actualidad de la web resultaba caro en
cuanto a tiempo y dedicación. Una alternativa representó la
suscripción, copiando el sistema tradicional de revistas y
periódicos. Si estabas suscrito recibías el aviso de la página web
actualizada. El aviso sólo era eso: un aviso. Nada más. De nuevo un
problema aparecía en el momento de creer haber solucionado otro:
primero debíamos consultar el correo y, en caso de tener en la
bandeja de entrada la notificación, visitábamos la página
actualizada. El colmo de este planteamiento era que estas
notificaciones venían a engrosar el número de mensajes recibidos en
la bandeja de entrada.

La explosión del \emph{blogging} vino acompañada de una solución a
este problema. Se decidió adoptar un antiguo formato conocido por
el acrónimo RSS, que significa \emph{Really Simple Syndication}.
Este formato tenía unas características generales que resultaron
fundamentales para su uso generalizado y ampliamente adoptado:

\begin{itemize}
\item
  Formato estandarizado
\item
  Contenía el contenido de la noticia
\item
  Contenía metainformación como la fecha de publicación y autoría.
\end{itemize}
Aún conociendo las ventajas que suponía la utilización del RSS, es
todavía un acrónimo desconocido para la mayoría de internautas. En
general convive con la práctica de ir visitando las páginas, una
tras otra, para consultar las novedades de las páginas guardadas en
la carpeta \emph{Favoritos}.

\section{El lector de RSS}

Gracias a servicios de
\emph{democratización de la publicación en la web} como Blogger,
que facilitaron la creación de un blog en menos de tres clics de
ratón, se hizo patente la necesidad de un lector o herramienta que
gestionara las suscripciones RSS. Hubo tres planteamientos:

\begin{itemize}
\item
  La visión \emph{e-mail-centric}, es decir, como una funcionalidad
  nueva del cliente de correo y, por lo tanto, incrementaba la
  acumulación de información en dos frentes, el de la bandeja de
  entrada y el de las suscripciones.
\item
  La visión \emph{desktop-centric}, una aplicación de escritorio y,
  por lo tanto, sucedáneo del cliente de correo en el que guardar la
  información que recibimos de las suscripciones.
\item
  La visión \emph{web-centric}, un servicio web que aprovechara la
  naturaleza web del RSS. En esta dirección encontramos dos versiones
  diferentes de afrontar la organización y lectura de las
  suscripciones, la traducción del cliente desktop-centric a la web,
  como el malogrado Bloglines o el exitoso Google Reader, y la
  personalización de una pantalla de inicio que ofreciera al usuario
  una instantánea de las últimas entradas de las suscripciones.
\end{itemize}
\begin{quote}
The key is quality information, not quantity of information%
\footnote{Wilson, (2005).
\emph{\href{http://www.newscientist.com/article/mg18624973.400}{Info-overload harms concentration more than marijuana}}}

\end{quote}
Con el paso del tiempo, los usuarios más atentos a la
\emph{blogosfera} y a cualquier publicación de noticias
\emph{sindicable}, vieron cómo el número de suscripciones crecía y,
por ende, el volumen de noticias o \emph{feeds} diarias era cada
vez mayor, lo que provocaba que el lector RSS se colapsara con
relativa facilidad. Este exceso de información obligaba a los
incombustibles lectores de RSS a una lectura en diagonal, cuando
no, al abandono de gran parte de las noticias. Si parece ser un
escenario prematuro, no son pocas las voces que plantean desde hace
un tiempo el problema de la sobrecarga del flujo de información
entrante en el lector de RSS. El exceso de la información generada
por la publicación sin pausa de terceros, colapsa nuestra atención
e interés por el contenido que recibimos a diario.

\begin{quote}
Alguién me ha preguntado recientemente porqué hago dos post al día.
Es lo que me habéis pedido. En algún viaje, queriendo contar todo
lo que he podido ver en un evento, he intentado postear 4 y 5 posts
y enseguida he recibido un rumor de mis lectores. Oye, ¿Te crees
que eres el único que leo? ¿Que es eso de postear como una
ametralladora?%
\footnote{Hurst, Mark. (2007). \emph{Bit Literacy}. Good Experience Press.}

\end{quote}
No es infrecuente tomar alguna decisión cuando el volumen de
\emph{feeds} nuevos excede en cantidad a un límite razonable de
lectura. Pero, ¿en qué dirección tomarla? ¿eliminar las
suscripciones más promiscuas, aquéllas que publican con una
frecuencia de veinte o treinta noticias al día?, ¿filtrar las
noticias entrantes?, ¿borrar aquellas suscripciones que sólo
redundan en información, es decir, que actúan de eco de otras
publicaciones? El problema de filtrar la información entrante es
que, una noticia de enorme interés personal acabe detrás del filtro
y, por lo tanto, desaparezca. De hecho, el cliente web de RS más
utilizado, Google Reader, no incorpora ningún sistema o
funcionalidad que ayude a filtrar la información. En lugar de un
sistema de filtros, Google Reader apuesta por la indexación de los
\emph{feeds}, un sistema flexible de etiquetaje, una suerte de
\emph{blog} en el que el usuario publica aquellas noticias que
desea compartir o, finalmente, un sencillo panel que muestra unas
estadísticas básicas sobre nuestra actividad de lectura en relación
con los diferentes canales a los que estamos agregados. Antes la
abundancia \emph{recuperable} que un sistema inteligente de
filtrado.

\section{Feed Fatigue? You need a filter system!}

\begin{quote}
But the problem many people face is that there are so many sources
of information that we're trying to keep track of, we've become
buried. Information overload is a real problem for many web users,
and one way to cope with it is to filter your RSS feeds so you only
see what you want%
\footnote{Hallowell, E. M. (2005).
\emph{\href{http://tr.im/wSsS}{Overloaded circuits: Why smart people underperform}}}

\end{quote}
Filtrar de alguna manera el exceso de RSS no es un problema menor
si, como ya hemos apuntado más arriba, no se consigue diferenciar
aquello que \emph{resulta vital para mi} del ruido. Enfrentarse en
el momento actual al desarrollo de un sistema eficiente de filtrado
pasa por dos visiones, si bien a primera vista irreconciliables, sí
complementarias a medio plazo.

La primera visión pasa por la utilización de palabras clave que
actúen como barrera a cualquier noticia que no aparezca en su
contenido. Más sofisticado si cabe, pero no representa ninguna
novedad técnica, es la incorporación de expresiones regulares en
lugar de palabras clave. La aparición, hace un año, de Yahoo Pipes
causó una cierta expectación como algún que otro interrogante del
tipo \emph{¿para qué sirve?}. En un primer momento se vendió como
una herramienta web-centric de creación de \emph{mashups}%
\footnote{Malhotra, Naresh K. (1984).
\emph{\href{http://www.jstor.org/pss/2488913}{Reflections on the Information Overload Paradigm in Consumer Decision Making}}}.

Yahoo Pipes debe su nombre a un operador de Unix, la barra vertical
o \emph{tubería} que enlaza la salida y entrada de datos entre dos
comandos. Aunque no estemos familiarizados con la línea de
comandos%
\footnote{Jacoby, J. (1984),
\emph{\href{http://www.jstor.org/pss/2488912}{Perspectives on Information Overload}}},
la idea es muy simple: gracias a un conjunto de operadores
visuales, Yahoo Pipes actúa como una suerte de tubería entre la
fuente de información y el resultado final. Por ejemplo, el
operador \emph{RegExp} permite al usuario definir expresiones
regulares que actúen como filtro de los \emph{feeds} entrantes.
Luego Yahoo Pipes transforma una masa informe de ruido informativo
en un flujo de información adecuada y aproximada a los intereses
del usuario.

\begin{quote}
So where is the startup that is going to be my information filter?
I am aware of a few companies working on this problem, but I have
yet to see one that has solved it in a compelling way. Can someone
please do this for me? Please? I need help. We all do.%
\footnote{Schneider, Ursula. (2002).
\emph{\href{http://www.jucs.org/jucs_8_5/the_knowledge_attention_gap/Schneider_U.html}{The Knowledge-Attention-Gap: Do We Underestimate The Problem Of Information Overload In Knowledge Management?}}}

\end{quote}
Yahoo Pipes no es el único servicio orientado a filtrar la
sobrecarga de \emph{feeds} (aunque no sea su propósito). En cierta
manera, el alcance de Yahoo Pipes supera el simple ejercicio de
filtrar la saturación de \emph{feeds}. Otras empresas o
\emph{startups} ofrecen la oportunidad de rebajar el flujo
informativo a través de la personalización de filtros: FeedRinse%
\footnote{Barthes, Roland. 1980. \emph{La cámara lúcida}. Paidós};
sin excesiva complejidad, Feedsifter%
\footnote{Italo Calvino, \emph{Las ciudades invisibles}}
permite crear \emph{in situ} el filtro de una fuente mediante
palabras clave.

\section{La conversación como filtro}

\begin{quote}
I need less data, not more data. I need to know what is important,
and I don't have time to sift through thousands of Tweets and
Friendfeed messages and blog posts and emails and IMs a day to find
the five things that I really need to know%
\footnote{\href{http://en.wikipedia.org/wiki/Personal_information_management}{http://en.wikipedia.org/wiki/Personal\_information\_management}}

\end{quote}
Gran parte de la actividad en la conversación es compartir
información. La mayoría de las herramientas de la conversación se
fundamentan en esta actividad que, en algunos momentos, resulta
frenética. No es extraño que, con frecuencia, se produzca el
fenómeno de \emph{echo chamber} o una reverberación de la
información debido al efecto de compartir la misma información
desde diferentes canales y usuarios. Compartimos una noticia en
Twitter o FriendFeed y, al poco tiempo, la misma noticia rebota
literalmente en la conversación, luego nos llegará publicada como
artículo en otro blog, como marcador, como \emph{retweet} del
nuestro, como comentario en FriendFeed, como noticia compartida en
Google Reader. La reverberación de una información puede terminar
transformándose en una auténtica caja de grillos digitales (nace un
\emph{trend}). La muerte del cantante mundialmente popular Michael
Jackson inundó por completo la conversación. No es tan interesante
la noticia en sí misma -el fallecimiento inesperado del músico-
sino cómo esta información fue interpretada por la conversación.
Millones de voces publicaron su opinión del acontecimiento, pero
otras manifestaron su propia visión en relación al perfil de
identidad digital. Desde la recuperación de viejos juegos de
\emph{8 bits}, donde aparece un pixelado Michael Jackson, hasta un
perfil inventor -desconocido para todos- del cantante, cada voz de
la conversación se apropió de la noticia y la devolvió a la
conversación pasada por un tamiz único y personal. Esta hábil
reinvención, parte de remezcla colectiva y creación individual, de
la información en la conversación desplaza un modelo único de
difusión de contenidos. Toda voz es susceptible de ser un emisor y,
a un mismo tiempo, peculiar filtro de aquello que comparte, mezcla
de unicidad -gustos y aficiones particulares- y colectividad -actúa
de \emph{bridge} entre la noticia y el conjunto de redes sociales
al que pertenece.

Cada voz un filtro. Dejo de suscribirme a canales de información:
prefiero \emph{seguir} aquellas voces que actúan a un mismo tiempo
de filtro y emisores de información \emph{relevante},
\emph{auténtica}. La atención que pueda dedicar a un contenido será
proporcional al número de \emph{retweets} de las voces que soy
\emph{follower}. El viejo RSS, como el e-mail, ya se le identifica
como parte activa del ruido y sobrecarga de información actual. Los
filtros están dejando paso a opciones difusas: \emph{Share it},
\emph{I like it} o \emph{tweet me}. Si bien tampoco están
resultando una solución definitiva, ni siquiera parcial
(\emph{overtweets}, \emph{tweetspammers} y \emph{overlikeit}), sí
en cambio puede representar el principio de un nuevo horizonte en
el que puedan recortarse siluetas de nuevas tecnologías de filtros
que discriminen el ruido, el \emph{spam informativo} que poco o
nada nos interesa.

\begin{quote}
Bringing all of this Web messaging and activity together in one
place doesn't really help. It reminds me of a comment ThisNext CEO
Gordon Gould made to me earlier this week when he predicted that
Web 3.0 will be about reducing the noise.%
\footnote{minid.net (2009). \emph{En casa de herrero, cuchillo de palo}
\href{http://www.minid.net/2009/06/17/en-casa-de-herrero-cuchillo-de-palo/}{http://www.minid.net/2009/06/17/en-casa-de-herrero-cuchillo-de-palo/}}

\end{quote}
\chap{Kill Inbox}

\begin{quote}
I stopped using e-mail most of the time. I quickly realized that
the more messages you answer, the more messages you generate in
return. It becomes a vicious cycle. By trying hard to stop the
cycle, I cut the number of e-mails that I receive by 80 percent in
a single week%
\footnote{Blair, Ann. (2003).
\emph{Coping with Information Overload in Early Modern Europe}}

\end{quote}
Luis ha tomado una decisión drástica: ha comunicado a los
compañeros y compañeras de la empresa que va a dejar de utilizar el
correo electrónico como herramienta de comunicación. Expresa la
necesidad de recuperar su productividad personal. Considera que
pasar parte de la mañana en leer y responder correo es una
actividad improductiva: requiere un enorme esfuerzo de
concentración, especialmente si atiende la totalidad de los
mensajes, recibidos y pendientes de respuesta. Leer y responder
correo no es el problema, puntualiza Luis. El problema se encuentra
en las prácticas tóxicas que contaminan y ofuscan la bandeja de
entrada. El resultado es un cajón de sastre donde conviven, entre
muchos, mensajes importantes, reenvíos, adjuntos de archivos,
recordatorios, agradecimientos, publicidad, \emph{spam}, etc. La
determinación de Luis se realizó en dos fases. La primera fue
comunicar su deseo de abandonar la comunicación por e-mail. Recibió
respuestas de las personas con las que regularmente mantenía
correspondencia. La mayoría reaccionaron positivamente. Se
identificaban plenamente con la situación que planteaba Luis. La
pregunta que vino después fue conocer la manera de comunicarse con
él a partir del momento que apagase el cliente de correo.

Luis es experto en nuevas tecnologías. Se llama a sí mismo
\emph{social computing evangelist}. Cree firmemente que las nuevas
formas de comunicación en la web tendrán un papel decisivo en la
metodología de trabajo en equipo. Luis se defiende argumentando que
él no desea en ningún momento estar incomunicado. Todo lo
contrario, desea utilizar alternativas de comunicación más
eficientes que el e-mail. Los blogs, las wikis, la mensajería
instantánea (IM), la videoconferencia, las redes sociales son
ejemplos de alternativas al uso indiscriminado del correo
electrónico. Incluso recupera un hábito que parecía olvidado,
levantarse de la silla y dirigirse personalmente a otro compañero
de unidad en lugar de enviarse mensajes. Por lo tanto, Luis animó a
los demás a incorporar estas tecnologías en lugar del e-mail. Esta
acción representaba la segunda fase, la más difícil, puesto que
representaba una disrupción en la metodología de intercambiar
información en el ámbito profesional. En primer lugar, Luis expuso
que el e-mail no era un cliente de mensajería instantánea, no era
tampoco un \emph{gestor de tareas}, ni el mejor lugar para los
\emph{recordatorios} de reuniones, ni mucho menos un
\emph{sistema de control de versiones} de la documentación cuando
ésta era desarrollada por varias personas. Este amplio abanico de
funcionalidades sólo conseguía un desastroso efecto: los correos
relevantes quedaban ocultos entre todo lo demás.

\section{Aciertos y limitaciones del e-mail}

La rápida adopción del correo electrónico a finales de la década de
los noventa por particulares, organizaciones, empresas tuvo un
efecto desconocido: los mensajes se acumulaban en la bandeja de
entrada y su gestión diaria representaba una buena dosis de
esfuerzo y dedicación. No sólo eran mensajes entre particulares:
aparece el \emph{spam} (correo basura), el \emph{phising}, el
\emph{e-mail spoofing} y otras actividades maliciosas que irrumpían
en el flujo del correo. Aún así, la facilidad de uso y su
instantaneidad convierten el e-mail en el espacio central de la
comunicación y gestión de información personal, por lo que los
usuarios adjudican funcionalidades o servicios tanto a la cuenta
como al cliente de correo:

\begin{itemize}
\item
  \textbf{gestor de proyectos y documentos}. Cuando colaboramos en un
  proyecto o en la redacción de un documento en el que participan
  varias personas, el cliente de correo se transforma en un sistema
  de control de versiones.
\item
  \textbf{gestor de marcadores}. Enviamos mensajes con enlaces de
  interés personal o profesional o relacionado con un proyecto del
  tipo \emph{Mira cuando puedas esto\ldots{}} o
  \emph{¿Tienes presente esta página?}.
\item
  \textbf{calendario}. Recibimos mensajes del tipo
  \emph{El 28 de enero tendrá lugar\ldots{}} o
  \emph{Te recordamos que el próximo 28 de enero tendrá lugar}.
  También es generalizado el uso del e-mail como gestor de citas,
  reuniones o conferencias:
  \emph{Mañana reunión a las 11:00 en la sala\ldots{}” o “no olvides que mañana hemos de vernos para\ldots{}}
\item
  \textbf{gestor de contactos}. La acumulación de mensajes de un
  cliente proporciona un tipo de información que, de otra manera,
  resultaría imposible de localizar. Ejemplo de esto sería el número
  de teléfono, qué clase de proyectos está involucrado o una
  información más difusa, pero no menos importante, como son las
  curiosidaes, intereses personales y profesionales. Una de las
  parcelas que, en este sentido está siendo de utilidad es lo que se
  conoce por \emph{e-mail mining} o \emph{data mining e-mail} y que
  tiene como objetivo detectar patrones en el contenido de los
  mensajes.
\item
  \textbf{gestor de tareas}. Nos reenviamos mensajes del tipo
  \emph{Urgente terminar\ldots{}} o \emph{Escribir mañana a\ldots{}}.
\item
  \textbf{gestor de copias de seguridad}. No nos sentimos seguros de
  la fiabilidad del disco duro del ordenador, tampoco del lápiz USB,
  luego nos reenviamos una copia del documento en forma de adjunto.
  También esta será la práctica para transportar información de un
  ordenador a otro.
\item
  una \textbf{red social} en el que compartimos las fotos de
  vacaciones, diapositivas divertidas (o no tan divertidas),
  chismorreos, rumores, causas y un largo etcétera.
\item
  un \textbf{diario personal} en el que vamos registrando nuestras
  memorias.
\end{itemize}
La sobredimensión del e-mail no tardará en ser objeto de estudio.
La bandeja de entrada llena es sinónimo de frustración y estrés
profesional. La acumlación de centenares o miles de mensajes es una
obsesión por la preservación de una suerte de copia de seguridad de
la memoria personal. Veamos a continuación un repaso a la
literatura de \emph{autoayuda} sobre la bandeja de entrada llena.

\section{El zen de una bandeja de entrada limpia}

Varios autores, unos más populares que otros, han prestado su
atención al problema del descontrol de la bandeja de entrada.
Merlin Mann, ideólogo de \emph{Inbox Zero}, Tim Ferriss, autor de
\emph{The 4-Hour Workweek}, Mark Husrt, de \emph{Bit Literacy},
Scott Young, en \emph{e-mail Zen}, incluso cabría el tremendamente
popular \emph{Getting Things Done} de David Allen, y otros
activistas \emph{zen} de la productividad, proponen consejos,
ayudas, recomendaciones, orientaciones e incluso qué herramientas
son las más eficientes para alcanzar ese estadio de equilibrio
espiritual moderno representado por una bandeja de entrada limpia,
controlada. En general, coinciden en que una bandeja de entrada
desbordada, un cliente de correo desbocado, supone un triple
problema:

\begin{itemize}
\item
  Se acumula el trabajo pendiente -\emph{e-mail cuttering}-. Un larga
  lista de mensajes no leídos es motivo de estrés y de disminución de
  eficiencia al obligarnos a dedicar gran parte de la jornada laboral
  a gestionar la bandeja de entrada.
\item
  Una creciente dispersión de la información
  -\emph{e-mail scattering}-. El contenido de los mensajes se pierde
  en la confusión de una bandeja de entrada desbordada. Recuperar más
  tarde un mensaje puede costarnos un tiempo y esfuerzo excesivo.
\item
  Una angustia personal -\emph{e-mail suffering}- al no conocer una
  manera mejor y eficiente de gestionar la bandeja de entrada.
\end{itemize}
Esta literatura colinda con la autoayuda profesional y los autores
acostumbran a adoptar una postura agnóstica en términos de qué
cliente es mejor o más idóneo para rebajar tensiones y ser más
eficientes en el trabajo.

En general, los diferentes autores coinciden en mayor o menor grado
en la siguiente lista de recomendaciones:

\begin{itemize}
\item
  Utilizar un cliente de correo eficiente que filtre lo máximo
  posible el correo basura, spam. Publicidad, basura, información
  indiscriminada o \emph{junk}: basura o no, la mayoría de estos
  mensajes sólo consiguen distraernos del resto de mensajes en cuanto
  a su cantidad. Todos los autores coinciden en no abrir jamás estos
  mensajes y borrarlos inmediatamente.
\item
  No actualizar la bandeja de entrada cada momento.
  \emph{No trabajas en un silo de misiles nucleares en Corea del Norte},
  como remarca Merlin Mann.
\item
  Concentrar en una única bandeja de entrada en caso de utilizar dos
  o más cuentas de correo.
\item
  Utilizar un cliente de correo web frente a uno de escritorio.
  Evitará de esta manera futuras migraciones. Independiente del
  sistema operativo, lugar.
\item
  Utilizar teclas de acceso rápido para ser más eficientes con el
  cliente de correo.
\item
  Concentrar la lectura del correo en un momento del día. No
  dispersar esta acción a lo largo del día.
\item
  Evitar leer el correo desde dispositivos móviles si no quieres ser
  un esclavo de la bandeja de entrada las 24 horas del día, 7 días a
  la semana.
\item
  Responder y borrar un mensaje que no necesite más de 1 minuto la
  redacción de la respuesta.
\item
  Cancelar subscripciones a listas de correo que ya no les prestamos
  ninguna atención. Si sigue interesándonos, agrupar los mensajes por
  días.
\item
  Borrar inmediatamente los mensajes que contengan causas colectivas,
  presentaciones de dudosa procedencia, chistes u otro material
  improcedente.
\item
  Penalizar a quien nos envía mensajes regularmente con contenido
  improcedente. Convencerse que un cliente de correo no es una
  herramienta con el que registramos la lista de tareas personal o
  del trabajo, guardamos los marcadores web, gestionamos las
  versiones de los documentos, un sistema de copias de seguridad de
  documentos o un cliente de mensajería instantánea.
\item
  Evitar el envío indiscriminado de mensajes ya que es probable que
  recibas un número igual o mayor de respuestas.
\item
  Borrar automáticamente todo mensaje pasado un tiempo (1 mes, por
  ejemplo).
\item
  Evitar posponer una respuesta más allá de 3 días. Pasado este
  tiempo es preferible borrar el mensaje.
\item
  Escribir menos, conciso y hacer un uso intensivo de listas de
  apartados como cuerpo del mensaje.
\item
  Contextualizar el contenido en el campo del título del mensaje.
\item
  Evitar la prosa, largas redacciones o florituras estilísticas. El
  receptor no espera leer la primera parte de una novela
  decimonónica.
\item
  Evitar adornos visuales, tipografía vistosa o colores que no sean
  armónicos en el cuerpo del mensaje. Utilizar alternativas de
  comunicación más efecientes si el medio (administración, empresa,
  compañía) lo permite.
\end{itemize}
\section{Dos visiones confrontadas: \emph{e-mail-centric} vs.\emph{stop-using-e-mail}}

Al margen de los buenos consejos -que acostumbran a funcionar sólo
a su dueño-, la tensión del e-mail no descansa en una gestión más o
menos eficiente de la bandeja de entrada. Como ya hemos adelantado
más arriba, dos son las tendencias que dibujan el futuro del correo
electrónico como sistema de comunicación. Una primera,
\emph{e-mail-centric}, que integra definitivamente aquellas
funcionalidades que el usuario ha incorporado al margen del primero
y único uso del cliente de correo, recibir, leer y enviar correo.
La segunda visión propone, como la iniciativa de Luis Suárez, la
eliminación por completo del correo electrónico. Porque, en
definitiva, el e-mail
\emph{no forma parte de la conversación actual} y es más un
incordio que un sistema válido de comunicación.

\subsection{GMail o la visión \emph{e-mail-centric} llevada al límite.}

\begin{quote}
We didn't want to simply bolt new features onto old interfaces. We
needed to rethink e-mail, but at the same time we needed to respect
that e-mail already had over 30 years of history, thousands of
existing programs, and nearly a billion users%
\footnote{Wilson, (2005).
\emph{\href{http://www.newscientist.com/article/mg18624973.400}{Info-overload harms concentration more than marijuana}}}

\end{quote}
O lo que vendría a ser un cliente de correo \emph{on steroids}. Su
aparición, en abril del 2004, despertó un mayor interés en los
\emph{early-adopters} que en el seno de la empresa, que dudaron y
se resistieron%
\footnote{Hurst, Mark. (2007). \emph{Bit Literacy}. Good Experience Press.}
al desarrollo de la nueva tecnología, pensando en la poca
penetración que tendría en un sector dominado por el cliente de
correo de escritorio de Microsoft, Outlook, o el popular Hotmail
(ahora MSN Hotmail). Un sistema de invitaciones, un buscador marca
de la casa, capacidad, buenas dosis de tecnología AJAX
(\emph{Asynchronous Javascript and XML}) y un nuevo método de
organización y clasificación de los mensajes, la vista de
conversación (emulando los mensajes de un foro) y el uso etiquetas
en lugar del tradicional sistema de carpetas, contribuyó que GMail
se convirtiera, en poco tiempo, en el principal cliente de correo.

Desde entonces, el \emph{laboratorio} del equipo de desarrollo de
GMail no ha descansando en ampliar los límites de la reinvención
del e-mail. La integración de una mensajería instantánea y, más
tarde, de videoconferencia, podía interpretarse como una estrategia
para desplazar la administración de la bandeja de entrada a la
conversación actual y reducir, en la medida de lo posible, el envío
de mensajes. Una suerte de \emph{metacliente} de correo que anima a
sus usuarios a no confundir la bandeja de entrada con la
conversación actual. El cliente de correo deja atrás la imagen de
aplicación monolítica y, siguiendo los pasos de la evolución del
móvil -de simple telefóno \emph{sin cables} a un pequeño, pero
potente ordenador de bolsillo-, deviene un sistema poliédrico de
gestión de información personal. Se le da al usuario lo que pide:
gestor de tareas, calendario, \emph{disco duro}, agenda,
sincronización con diferentes dispositivos, notificaciones
asíncronas%
\footnote{Hallowell, E. M. (2005).
\emph{\href{http://tr.im/wSsS}{Overloaded circuits: Why smart people underperform}}}
y un largo etc. que asegura un vida prolongada del \emph{primitivo}
sistema.

\subsection{\emph{stop-using-e-mail}}

\begin{quote}
Esta es la primera crisis fuerte en la que disponemos masivamente
desplegada de esta tecnología de comunicación electrónica. Lo que
me dicen y lo que puedo observar es que no está resultando todo lo
efectiva, productiva y humana que pensaron sus inventores, como
explicó uno de ellos, Howard Frank, cuando le dimos un premio en el
IESE hará 15 años. Hay quienes se esconden y evitan disculparse.
Pero también los hay creando la imagen de que trabajan reduciendo
costes cuando lo que hacen es superficial o contraproducente. Casi
seguro que para la próxima crisis fuerte, el 2021, los e-mail
estarán prohibidos%
\footnote{Malhotra, Naresh K. (1984).
\emph{\href{http://www.jstor.org/pss/2488913}{Reflections on the Information Overload Paradigm in Consumer Decision Making}}}

\end{quote}
La batalla personal de Luis Suárez contra la acumlación de correo
en la bandeja de entrada no acabó en una simple declaración de
buenos propósitos. Transcurrido un año, Luis sigue evaluando la
iniciativa de aparcar el cliente de correo como canal de
comunicación principal con el resto de compañeros de trabajo%
\footnote{Jacoby, J. (1984),
\emph{\href{http://www.jstor.org/pss/2488912}{Perspectives on Information Overload}}}.
La variedad de alternativas al uso indiscriminado del e-mail son
cada vez más numerosas y, sobretodo, factibles. Redes sociales,
servicios de microblogging o cualquiera de los clientes de
mensajería instantánea o VoIP representan, en la actualidad,
herramientas de uso generalizado en el ámbito particular. Un
ejemplo de esto representa la diversidad de medios de comunicación
utilizados por los ridículamente bautizados como
\emph{digital natives} o \emph{net natives}. Estos se inclinan por
tecnologías de mensajería instantánea, redes sociales, SMS (Short
Message Service) o, simplemente, teléfono móvil y sólo utilizan el
correo electrónico en entornos rehacios a formar parte de la
cultura de la conversación, como el caso del ecosistema de la
administración pública y, en especial, el mundo educativo. La
enorme interacción de muchos a muchos coloca la dinámica del e-mail
en una posición de clara desventaja. Quizá una
evolución/transformación del e-mail en esta dirección sea el
\emph{comentario}, asociado a cualquier evento que publicamos, sea
una entrada del blog, un \emph{tweet} o una actividad o estado en
la red social.

\begin{quote}
No E-Mail Account: You are in your late teens or early 20s and you
equate sending e-mails with using a fax machine, watching broadcast
TV or buying CDs --- lame. You text and/or IM, and that's it. TTYL%
\footnote{Schneider, Ursula. (2002).
\emph{\href{http://www.jucs.org/jucs_8_5/the_knowledge_attention_gap/Schneider_U.html}{The Knowledge-Attention-Gap: Do We Underestimate The Problem Of Information Overload In Knowledge Management?}}}

\end{quote}
La posibilidad de alcanzar un éxito en abandonar definitivamente el
e-mail es limitada si el contexto en el que se desarrolla es de
ámbito profesional y corporativo. El celo con el que empresas y
corporaciones gestionan la comunicación interna de sus empleados
acaba en la penalización de la utilización de tecnologías externas
-fuera de los servidores corporativos- como el enquiste de la
ineficencia y baja calidad de las propias. Si el lamentable
Internet Explorer 6 todavía sigue utilizándose mayoritariamente%
\footnote{Barthes, Roland. 1980. \emph{La cámara lúcida}. Paidós}
por estas y otras políticas restrictivas en el equipamiento
informático de administraciones públicas y empresas%
\footnote{Italo Calvino, \emph{Las ciudades invisibles}},
no resulta difícil extrapolar esta delirante situación al
tratamiento del correo:

\begin{itemize}
\item
  acumulación desorbitada de correo basura o malicioso por falta de
  una tecnología eficaz del proveedor contra el spam.
\item
  limitaciones ridículas de espacio de la bandeja de entrada como del
  tamaño de transferencia de los
  mensajes.\verb!Action: failed Status: 4.2.2 (user over quota; cannot receive new mail)!
\item
  inseguridad y dudosa efectividad si urge la necesidad realizar
  copias de seguridad del correo
\item
  dificultad en la localización de contenido específico a través del
  buscador interno del propio cliente de correo.
\end{itemize}
El e-mail, tal como lo conocimos a mediados de la década de 1990,
no forma parte de la conversación actual. Su presencia ubicua en el
ecosistema digital no corresponde con la dinámica actual de
compartir información. Todo lo contrario, representa uno de los
generadores de mayor ruido y basura no visible debido a su carácter
privado. Al margen de sus efectos nocivos en términos de
productividad, resulta evidente la necesidad de frenar su
prolongación en el tiempo. Google invierte esfuerzos en las dos
direcciones al dotar, por un lado, a su popular cliente de correo
GMail, funcionalidades que reinventan la gestión de la bandeja de
entrada y, por otro, diseñar un entorno híbrido de comunicación o
\emph{mashup} a tiempo real que permita la personalización
individual y colectiva de la conversación%
\footnote{\href{http://en.wikipedia.org/wiki/Personal_information_management}{http://en.wikipedia.org/wiki/Personal\_information\_management}}.

\begin{quote}
What's wrong with e-mail? In a nutshell, the medium is perfectly
designed for information overload. Both message size and quantity
are essentially unlimited. Unfortunately, electronic communication
is like a gas: It expands to fill its container%
\footnote{minid.net (2009). \emph{En casa de herrero, cuchillo de palo}
\href{http://www.minid.net/2009/06/17/en-casa-de-herrero-cuchillo-de-palo/}{http://www.minid.net/2009/06/17/en-casa-de-herrero-cuchillo-de-palo/}}

\end{quote}
\chap{The wiki old days}

En la elaboración de un documento en el que participa más de una
persona se establece, en general, la metodología de trabajo de
divide y vencerás. Cada responsable acepta una parte de la
redacción del contenido. De esta forma, nadie pisa el trabajo del
otro. Esta modalidad también incluye un baremo de evaluación del
trabajo de cada uno. El trabajo final es una extraña suerte de suma
de las partes. Si a primera vista esta metodología nos parece
idónea, un análisis un poco más detallado nos revela que:

\begin{itemize}
\item
  Cada responsable desconoce la evolución del trabajo del resto del
  equipo.
\item
  Los responables deben coincidir en utilizar tecnologías idénticas
  que permitan unificar el estilo visual del documento.
\item
  Se crea la dinámica del \emph{archivo adjunto} como gestión de las
  versiones del documento.
\item
  Resulta muy difícil la tarea de coordinar el desarrollo del
  documento debido fundamentalmente a la fragmentación del contenido.
\item
  Hay tantas copias o más del contenido como responsables del mismo.
\end{itemize}
El problema fundamental se manifiesta cuando quienes participan en
el desarrollo del documento intentan revisar cambios anteriores.
¿Quién ha realizado este o aquel cambio? ¿Cuándo se realizó la
modificación? ¿Podemos recuperar una versión anterior del documento
previo a la modificación? Y una vez terminada la redacción, ¿quién
y cómo se encargarán las futuras actualizaciones del documento?
Este escenario se agrava más todavía si una metodología de
desarrollo de la documentación personal y, especialmente,
colectiva, no permite de manera transparente la reutilización de
los contenidos. Es probable que un mismo contenido se escriba más
de una vez en organizaciones en las que las unidades que la
conforman generan la documentación en la más pura de las
invisibilidades; se hablaría de unidades estancas aunque las separe
físicamente apenas un metro de distancia entre ellas. La
duplicación y la triplicación de los contenidos tiene un efecto
desastroso en la productividad de la organización como manifestar
abiertamente las incoherencias internas. De hecho, es posible que
un único mensaje se represente, se escriba o se transmita de varias
maneras, debido a una visión centralista de la gestión de la
información.

\section{Wikis}

\begin{quote}
Nevertheless, I predict that Wikis will disappear over the next 5
to 10 years. This is not because they will fail but precisely
because they will succeed. The best technologies disappear from
view because they become so common-place that nobody notices them.
Wiki-style functionality will become embedded within other software
– within portals, web design tools, word processors, and content
management systems. Our children may not learn the word ``Wiki,''
but they will be surprised when we tell them that there was a time
when you couldn't just edit a web page to build the content
collaboratively%
\footnote{Blair, Ann. (2003).
\emph{Coping with Information Overload in Early Modern Europe}}

\end{quote}
Según la Wikipedia, una wiki es una o más páginas web que permite a
los usuarios modificar su contenido utilizando un lenguaje de
marcado simplificado a través de un navegador. La wiki más popular
es la Wikipedia, una enciclopedia libre y sin ánimo de lucro
creada, actualizada y mantenida por voluntarios de todo el mundo.

La wiki surge a mediados de la década de 1990 y el próposito
original era, según en palabras de su creador, Ward Cunningham%
\footnote{Wilson, (2005).
\emph{\href{http://www.newscientist.com/article/mg18624973.400}{Info-overload harms concentration more than marijuana}}}:

\begin{itemize}
\item
  invitar a todos los usuarios a editar cualquier página o crear
  nuevas dentro de la wiki, utilizando un navegador web.
\item
  promover la asociación de diferentes contenidos gracias a la
  creación intuitiva y fácil de enlaces entre páginas.
\end{itemize}
La adopción de la tecnología wiki fue rápida en gran medida por la
facilidad de publicar contenidos en la web sin necesidad de pasar
por un editor de HTML. Los usuarios sólo debían aprender un
sencillo lenguaje de marcado que identifica las cabeceras de
diferente nivel, la negrita, la cursiva, los enlaces, etc. de la
página. Este planteamiento favoreció la agilidad en el desarrollo
colectivo de la documentación, al situar el núcleo de trabajo en un
espacio único y editable en la web, accesible desde cualquier
ordenador que tuviera conexión y manejar un simple sistema de
control de versiones de las páginas del documento.

La siguiente lista presenta los argumentos más reiterados que se
esgrimen a favor de esta visión de las wikis como potentes
herramientas de productividad en la gestión documental:

\begin{itemize}
\item
  \textbf{Integradora}. Evita la dispersión de la documentación y la
  información relacionada a su gestión. Esta fragmentación se debe
  generalmente al uso de canales de comunicación como el e-mail, en
  forma de adjuntos que van y vienen , o calendarios de trabajo. Una
  wiki facilitará a los usuarios encontrar rápidamente la información
  sólo accediendo a la web de la wiki desde cualquier navegador.
\item
  \textbf{Gestión eficiente del histórico de la documentación}. Una
  wiki incorpora un sistema transparente de control de versiones,
  luego permitirá a los usuarios saber en cada momento quién, cuando
  y cómo ha contribuido, modificado o borrado. También resultará
  posible recuperar sin problemas versiones anteriores del documento.
\item
  \textbf{Centralizada}. En lugar de dispersar la documentación por
  los diferentes dispositivos de memoria de los usuarios, se unifica
  en un único lugar, la wiki.
\item
  \textbf{Visible}. Especialmente si nos referimos a que otros grupos
  o unidades de trabajo puedan acceder a la información
\item
  \textbf{Indexable}. Una wiki incorpora un sistema de indexación de
  los contenidos, luego permite que los usuarios realicen búsquedas
  internas de la documentación.
\item
  \textbf{Independiente}. Una wiki no está sujeta a un software
  específico y, por lo tanto, no es susceptible de ver el contenido
  afectado por futuras actualizaciones. Sólo necesita un navegador
  web, por lo que resultará también ajena a una exclusiva plataforma,
  sistema operativo o dispositivo.
\item
  \textbf{Coherente}. La unificación de la documentación en la wiki
  permite que el resultado sea más coherente en términos de estilo.
\item
  \textbf{Predominio del contenido sobre la forma}. Quienes editen la
  wiki sólo han de preocuparse de entrar contenidos y no vigilar si
  incumplen o alteran la plantilla visual del documento, como ocurre
  en general al trabajar con los procesadores de texto.
\item
  \textbf{Estructural}. La edición mediante un simple lenguaje de
  marcado favorece, en contra de los editores visuales, la estructura
  de los contenidos.
\item
  \textbf{Exportable}. En general, una wiki incluye un sistema
  transparente de exportación de los contenidos a otros formatos de
  documentación, como HTML y PDF.
\item
  \textbf{Gestión} de la edición y la visibilidad de los contenidos.
  Una wiki incorpora sistemas de control de permisos de acceso a los
  usuarios, luego es posible establecer reglas de lectura y escritura
  de las diferentes páginas.
\end{itemize}
\section{Lenguaje de marcado en la era de los editores visuales}

Una de las dificultades, que la mayoría de usuarios presentan como
barrera insuperable cuando se enfrentan por primera vez a una wiki,
es la redacción de texto mediante un lenguaje de marcado en lugar
del acostumbrado editor visual.

Cristoph Sauer ya expuso en
\emph{What you see is Wiki - Questioning WYSIWYG in the Internet Age}%
\footnote{Hurst, Mark. (2007). \emph{Bit Literacy}. Good Experience Press.}
cuáles son los puntos a favor y en contra de utilizar un lenguaje
de marcado frente a un editor visual en el caso de las tecnologías
wikis. En general, los puntos en contra de los lenguajes de marcado
son los siguientes:

\begin{itemize}
\item
  El hábito de editar el texto de manera visual, aquello que veo en
  pantalla es aquello que saldrá por la impresora, siguiendo la
  lógica de los editores visuales, como Microsoft Word o OpenOffice
  Writer, empuja a los usuarios a rechazar de pleno el trabajo con
  una wiki. Conocer un lenguaje de marcado representa poco menos de
  15 minutos de aprendizaje. Explicar la lógica de utilizar un
  lenguaje de marcado y sus beneficios en la creación de textos
  estructurados puede representar mucho más que los 15 minutos de
  conocimiento del lenguaje de marcado
\item
  El hábito de modificar libremente el estilo del documento en el
  caso de los editores visuales frente a la rígida separación de
  contenido y presentación de los lenguajes de marcado es otro de los
  problemas que los usuarios presentan como un motivo más para no
  adoptar una wiki.
\item
  La dificultad de navegar visualmente por el flujo del texto en los
  lenguajes de marcado es otro de los inconvenientes frente a una
  visualización del texto diferenciado según sea su formato.
\item
  La falta de una estandarización del lenguaje de marcado de las
  diferentes tecnologías wikis no contribuye a una rápida adopción
  por parte de los usuarios.
\end{itemize}
Estos argumentos acostumbran a ser los más utilizados en contra del
uso de un lenguaje de marcado. No representan un problema técnico,
sino de concepto. Los lenguajes de marcado priorizan la estructura
del contenido, luego el lenguaje de marcado no representa un
\emph{callejón sin salida}, no es el final del camino, sino la
\emph{fuente} del contenido para su represantación en otros
formatos (como por ejemplo, en PDF) que sí tienen carácter de
\emph{instantáneas} de la documentación. El lenguaje de marcado,
por otro lado, es sinónimo de flexibilidad: una misma fuente de
información puede representarse sin dificultad en cualquier
dispositivo o aplicación actual y futura gracias al desarrollo de
\emph{parsers}, que transforman fácilmente el lenguaje de marcado a
otra sintaxis (un ejemplo sería la conversión de wiki a HTML).

Por otro lado, si hablamos en el contexto de la web, cada vez son
más los usuarios que reelaboran los contenidos, sean propios o
ajenos. De hecho, en la mayoría de casos sólo interesa un fragmento
o parte del contenido. Si el contenido es dependiente de una
tecnología propietaria, si la presentación se mezcla con la
estructura, estaremos obligados a darle de nuevo un formato que se
ajuste al documento que estamos elaborando. Por el contrario, el
lenguaje de marcas de una wiki es texto crudo, plano, no está
asociado a ninguna tecnología ni a un formato visual, lo que
facilitará enormemente su portabilidad e interoperablidad entre
diferentes escenarios de desarrollo, especialmente el de la
remezcla.

\section{wikis corporativos ¿una realidad imposible?}

\begin{quote}
Nearly all enterprise wikis have implemented a WYSIWYG editor. Even
ones that began with MediaWiki, like Mindtouch, quickly replaced
wiki syntax with XHTML. The Initiative has attempted to draw on the
experience of companies like Mindtouch, whose CEO, Aaron Fulkerson,
told ReadWriteWeb he was impressed with the Initiative, but that he
felt, ``wiki text will always and forever be inferior to XHTML.''%
\footnote{Hallowell, E. M. (2005).
\emph{\href{http://tr.im/wSsS}{Overloaded circuits: Why smart people underperform}}}

\end{quote}
Si la utilización colectiva de una wiki ofrece claras ventajas en
la gestión y desarrollo de la documentación, no resulta tan
evidente su adopción en el ámbito corporativo:

\begin{itemize}
\item
  La edición siguiendo un lenguaje de marcado frente al hábito de
  trabajar con editores visuales es quizá el mayor inconveniente o
  problema que plantean los usuarios al enfrentarse por primera vez a
  una wiki. El hábito o costumbre de trabajar con procesadores de
  texto visuales representa un escollo a veces insalvable. Superar el
  rechazo al lenguaje de marcas ha llevado a que los actuales
  sistemas de edición en línea, a diferencia de las wikis
  tradicionales, incorporen un editor visual en lugar de un simple
  editor de etiquetas, como resulta evidente en el caso del
  procesador de textos de Google.
\item
  La falta de hábito en compartir la documentación, o trabajar de
  manera colaborativa en entornos de la administración o de la
  empresa, es otro de los grandes inconvenientes.
\end{itemize}
\begin{quote}
WYSIWYG editors are already available for MediaWiki through
extensions, but the potential for corrupting the data that makes up
Wikipedia's encyclopedic content is very real. Avoiding that
scenario is primarily what lead the Initiative to discount a switch
to WYSIWYG, at least within the scope of the project.%
\footnote{Malhotra, Naresh K. (1984).
\emph{\href{http://www.jstor.org/pss/2488913}{Reflections on the Information Overload Paradigm in Consumer Decision Making}}}

\end{quote}
\section{Reinventar la wiki ¿es posible?}

En el artículo \emph{How to build the perfect wiki}%
\footnote{Jacoby, J. (1984),
\emph{\href{http://www.jstor.org/pss/2488912}{Perspectives on Information Overload}}},
Tom Morris manifiesta abiertamente estar en contra de los sistemas
wikis actuales. Según Morris:

\begin{itemize}
\item
  cuando una wiki aparece bajo la etiqueta \emph{enterprise},
  inevitablemente piensa en un producto lo suficientemente
  complicado, intratable y lejos de poder ser \emph{hackeable} que
  por ello mismo sólo puede venderse como producto comercial.
\item
  trabajar con el editor \verb!<textarea />! del navegador web es
  frustrante. Si trabajar con los editores llamados \emph{visuales}
  de una wiki es terriblemente ineficiente (pervierten su sentido de
  estructura del contenido) al imitar la edición de los procesadores
  de textos, la \emph{otra} edición, que sólo incorpora un número
  limitado de botones de ayuda de sintaxis (negrita, itálica,
  secciones y subsecciones, subrayado, enlaces, imágenes,\ldots{})
  tampoco resulta ninguna panacea.
\item
  la navegación por el texto no es intuitiva (la fuente
  \emph{monospace}, de ancho fijo, no es la más indicada aunque nos
  recuerde que estamos \emph{editando}).
\item
  la operación de \textbf{búsqueda y reemplazo de texto} es
  inexistente en la mayoría de editores wiki.
\item
  en realidad, una wiki es
  \emph{un sistema de control de versiones de páginas web}, pero muy
  limitado si lo comparamos con sistemas de control de versiones como
  Subversion o Git.
\item
  ¿qué ocurre si queremos trabajar en modo \emph{offline}? La mayoría
  de wikis no permiten todavía trabajar desconectados.
\item
  la sintaxis sigue siendo un problema, pero desde la falta de un
  estándar de facto. En general, cada wiki funciona mediante una
  sintaxis propia, como si se tratara de dialectos de un mismo
  idioma%
  \footnote{Schneider, Ursula. (2002).
\emph{\href{http://www.jucs.org/jucs_8_5/the_knowledge_attention_gap/Schneider_U.html}{The Knowledge-Attention-Gap: Do We Underestimate The Problem Of Information Overload In Knowledge Management?}}}.
  Ni la aparición hace un tiempo de Creole%
  \footnote{Barthes, Roland. 1980. \emph{La cámara lúcida}. Paidós},
  un proyecto nacido con la buena intención de corregir esta
  dispersión, ha motivado el esperado consenso. Creole será a las
  wikis lo que el Esperanto a los idiomas.
\end{itemize}
En cambio, según Tom, la solución pasaría por trabajar en local y
utilizar un verdadero sistema de control de versiones en lugar de
los que incorporan los sistemas wikis, muy rudimentarios en
comparación. Lo que veríamos en la web correspondería al último
\emph{commit} o modificación del contenido. Los sistemas de control
de versiones están muy ligados al desarrollo de software. El uso de
una herramienta de control de versiones es imprescindible en el
desarrollo de cualquier software. A modo de introducción, una
herramienta de control de versiones ayudará a los desarrolladores a
gestionar de manera eficaz la evolución de un proyecto. como
realizar instantáneas, \emph{snapshots}, del código fuente en el
que se está trabajando, comprobar diferencias entre diferentes
instantáneas y revertir a una instantánea anterior si es necesario.
Tecnologías como CVS (Concurrent Version System), Subversion,
DARCS, Mercurial o Git son sistemas de control de versiones de
licencia GNU/GPL más populares y ampliamente utilizados. Sí
comparten mecanismos de trabajo muy parecido y no en el
planteamiento de la gestión y mantenimiento de las versiones del
proyecto. Desde un principio, los sistemas de control de versiones,
como CVS o Subversion, tuvieron una planteamiento fuertemente
centralizado, es decir, el \emph{repositorio}, el lugar donde el
sistema de control de versiones mantiene el registro de todos los
cambios (qué, quién y cuando), está ubicado en un único lugar o
servidor.

Frente a los \emph{repositorios centralizados} encontramos un
enfoque diferente: los \emph{repositorios distribuidos} o DVCS
(\emph{Distributed Version Control Systems}). La diferencia más
notable es que cada desarrollador tiene su propio
\emph{repositorio} en local, luego se tiene el acceso a toda la
historia del proyecto, examinar el histórico de cada archivo sin
depender o no de la conexión o acceso al repositorio. Sin entrar en
más detalle, esta visión \emph{distribuida} del sistema de control
de versiones se ha ido adentrando en el terreno de la conversación,
especialmente desde el nacimiento hace unos pocos años de Git%
\footnote{\href{http://en.wikipedia.org/wiki/Personal_information_management}{http://en.wikipedia.org/wiki/Personal\_information\_management}}
a cargo de Linus Torvalds, padre también de Linux, como de la mano
de GitHub%
\footnote{minid.net (2009). \emph{En casa de herrero, cuchillo de palo}
\href{http://www.minid.net/2009/06/17/en-casa-de-herrero-cuchillo-de-palo/}{http://www.minid.net/2009/06/17/en-casa-de-herrero-cuchillo-de-palo/}}.
El lema de GitHub se resume en los comentarios de Ryan Tomayko%
\footnote{\href{http://googleblog.blogspot.com/2009/05/more-search-options-and-other-updates.html}{More Search Options and other updates from our Searchology event, 2009}}

\begin{quote}
Forking on GitHub is like friending on Myspace (or Facebook or
whatever crazy ass social networking site that is) inasmuch as this
is the point where a line is drawn from one node to another in the
social graph.

\end{quote}
Incluso el lenguaje técnico se manifiesta abiertamente en la
conversación:

\begin{quote}
You want to ``friend me''? Send me a patch.

\end{quote}
Llegado a este punto no es difícil imaginar cómo la frustración de
Tom Morris con las wikis tradicionales ha encontrado una salida,
una extraña \emph{suerte} de sistema de publicación y gestión de
documentación gracias a la mezcla de la tecnología de Git y el
enfoque social de GitHub. Una simple búsqueda de la palabra clave
\verb!books! en GitHub no dice lo contrario%
\footnote{Mora, Vicente Luis. (2009)}.
La paradoja actual se muestra en toda su magnitud cuando un sistema
caduco de creación, edición y gestión documental se encuentra con
uno tan disparatadamente actual, eficiente y efectivo:

\begin{quote}
Somewhere else I read that someone liked that I used Markdown for
writing the book, as you can download the Markdown source for the
book at GitHub. Well, the entire writing process was unfortunately
not done in Markdown. At Apress most of the editing and review
process is still MS Word centric%
\footnote{Jones et al., (2006)}

\end{quote}
Incluso, de un tiempo a esta parte, se han realizado tímidos
intentos de llevar estos sistemas a ámbitos tan aparentemente
lejanos como la creación literaria%
\footnote{\href{http://www.apple.com/es/mobileme/}{http://www.apple.com/es/mobileme/}}:
el escritor y blogger Cory Doctorow utiliza un \emph{wrapper} de
Git, Flashbake%
\footnote{\href{http://www.pogoplug.com/}{http://www.pogoplug.com/}},
que en palabras de su creador

\begin{quote}
This project was inspired by Cory Doctorow asking me for
suggestions on source control for his writing and personal
information files.

\end{quote}
donde ya no sólo se reduce a manejar las versiones de cada
\emph{instantánea} del libro sino, como dice Doctorow, que el
propio sistema registre detalladamente el contexto personal del
autor mientras escribe el libro

\begin{quote}
Every 15 minutes, Flashbake looks at any files that you ask it to
check (I have it looking at all my fiction-in-progress, my todo
list, my file of useful bits of information, and the completed
electronic versions of my recent books), and records any changes
made since the last check, annotating them with the current
timezone on the system-clock, the weather in that timezone as
fetched from Google, and the last three headlines with your by-line
under them in your blog's RSS feed (I've been characterizing this
as
``Where am I, what's it like there, and what am I thinking about?'').
It also records your computer's uptime. For a future version, I
think it'd be fun to have the most recent three songs played by
your music player%
\footnote{\href{https://services.mozilla.com/}{https://services.mozilla.com/}}

\end{quote}
Este \emph{mashup} de pensamiento y tecnología, entre la creación
literaria y la tecnología, y unidas esta vez por la conversación,
quizás dibujen un enfoque completamente nuevo de cómo gestionar la
documentación, sea literaria o técnica, basada enteramente en la
idea de la \emph{remezcla}, del \emph{fork me}, del
\emph{send me a patch}. Para entonces la wiki tendrá una presencia
histórica, como pionera en la publicación colaborativa en la web,
pero habrá desaparecido definitivamente como tecnología.

\chap{Look Ma! Without tabs.}

Oliver Reichenstein%
\footnote{Blair, Ann. (2003).
\emph{Coping with Information Overload in Early Modern Europe}},
de Information Architecs%
\footnote{Wilson, (2005).
\emph{\href{http://www.newscientist.com/article/mg18624973.400}{Info-overload harms concentration more than marijuana}}},
publicaba una entrada en el blog que recordaba una pregunta que le
formularon a principios del año 2000:
\emph{Si tuvieras que diseñar la interfaz de un navegador desde cero ¿qué decisiones tomarías?}.
La respuesta se limitó a una lacónica respuesta: \emph{tabs!}
(pestañas). En realidad manifestaba un error de diseño flagrante de
la gestión de las páginas activas del navegador más utilizado en
aquel momento, Internet Explorer. Cada página nueva significaba una
ventana nueva del navegador. El efecto \emph{cluttering} estaba
asegurado al poco tiempo de navegar. Tampoco una mayoría de
\emph{diseñadores y programadores} de la web no ayudaron a
disminuir la saturación de ventanas del navegador por la limitada
extensión del escritorio de los usuarios. Todo lo contrario.
Obviaron -y siguen obviando- las flechas de navegación -como si
fueran un elemento decorativo- alegando como excusa la estupidez
del navegante. Surge el \emph{spam} de la navegación de la web, los
\emph{pop-ups}, ubicuas e intrusivas vallas publicitarias de
contenido basura. El escritorio atiborrado de ventanas abiertas o
minimizadas comenzó a ser una imagen familiar para los usuarios.
Las pestañas vinieron con el propósito de agrupar la dispersión de
la información: las páginas activas de una sesión de trabajo
estarían encerradas en los límites físicos del navegador; la
posición de las mismas en el horizonte visual facilitaría un rápida
localización de cada una de las páginas abiertas. La adopción del
sistema de pestañas por parte de los diferentes navegadores
representa una muestra de las irregularidades del desarrollo de
\emph{software}. Si las primeras apariciones datan de finales de la
década de 1990 (Ibrowse en 1999, Opera en el 2000 y Mozilla en el
2001) a mano de navegadores minoritarios, el popular y
monopolístico Internet Explorer de Microsoft tendrá que esperar
hasta 2007 (una muestra evidente de cómo los cambios suelen ser más
difíciles de ejecutar en empresas y corporaciones grandes que no en
pequeños equipos).

Ocho años después, Oliver Reichenstein responde de nuevo a la misma
pregunta, esta vez planteada por Aza Raskin%
\footnote{Hurst, Mark. (2007). \emph{Bit Literacy}. Good Experience Press.}
y responde con contundencia: \emph{forget tabs!} (olvida las
pestañas!). ¿Regresar al pasado? No. La situación ha cambiado mucho
desde entonces. El sistema de pestañas ha dejado de ser una
respuesta eficiente. La acumulación de pestañas es un problema y
las páginas actuales han dejado atrás su condición \emph{estática};
en la actualidad manejamos aplicaciones
\emph{encerradas en el cuerpo del navegador}. En la presentación de
Chrome%
\footnote{Hallowell, E. M. (2005).
\emph{\href{http://tr.im/wSsS}{Overloaded circuits: Why smart people underperform}}}
-el navegador de Google- a cargo del dibujante de cómics, Scott
McCloud, el \emph{product manager} de Google, Brian Rakowski,
sentencia en la primera viñeta que

\begin{quote}
Today, most of what we use the web for on a day-to-day basis aren't
just web pages, they are \textbf{applications}

\end{quote}
La web se ha vuelto compleja en comparación a la de hace una
década, pero, paradójicamente, nunca como ahora la web ha
prolongado los límites del escritorio personal hasta la \emph{nube}
de ordenadores. Esta construcción de la web obliga al navegador a
trabajar como una suerte de sistema operativo. El ejercicio de
gestionar eficazmente cada una de las aplicaciones web en cuanto a
rendimiento, seguridad y estabilidad representa la meta de los
navegadores actuales. Cada aplicación abierta, cada \emph{app-tab},
representa un intenso de consumo de recursos del sistema. En otras
palabras, el navegador devora al sistema operativo si hay un número
elevado de pestañas abiertas. Reichenstein cita jocosamente a Dios%
\footnote{Malhotra, Naresh K. (1984).
\emph{\href{http://www.jstor.org/pss/2488913}{Reflections on the Information Overload Paradigm in Consumer Decision Making}}}
cuando habla que las
\emph{pestañas no funcionan cuando la información es heterogénea}.
Si las pestañas, como sistema de organización de la información del
navegador, solucionaron el problema del \emph{cluttering} de
ventanas, veamos qué solución encontramos a la diversidad de
información que maneja el navegador. En palabras de Reichenstein,

\begin{quote}
The idea is not to show screen shots but to turn the browser into a
media system organizer more than a media display application.
Instead of structuring a browser to keep the screen tidy for the
moment, we thought that it'd be awesome to structure the browser as
a (multi media) file system.

\end{quote}
O dicho de otra manera, transformar el navegador en una suerte de
\emph{iTunes de la web}. Pero mientras esperamos el cambio que
jubile definitivamente la gestión de las páginas y aplicaciones web
por pestañas, otro modelo de interacción entre la web y el usuario,
menos \emph{browser-centric}, corresponde a la llamada
\emph{web contextual} o, en otras palabras, el uso de diversas
estrategias que aumenten la experiencia del usuario al visitar una
página web y eviten, indirectamente, el problema del exceso de
pestañas. Veamos cómo.

\section{La web contextual}

El significado de la web contextual es simple: cada página web no
es un contenido aislado. Todo lo contrario, el contenido define de
manera implícita un marco más amplio de información; una suerte de
capas de información asociadas a partes específicas y, por lo
tanto, relevantes, de cada página. Esta \emph{web aumentada}
permite mejorar la eficacia de algunas de las prácticas habituales
en la red:

\begin{itemize}
\item
  Mejorar la eficacia de la doble actividad \emph{buscar/guardar}
  información de la página web. La información contextual actúa de
  filtro en la búsqueda de información asociada a un contenido.
\item
  Mostrar el grado de participación social de la página web y, por lo
  tanto, conocer su grado de relevancia.
\item
  Incrementar la interoperabilidad entre diferentes páginas y
  aplicaciones web.
\end{itemize}
Lo anterior no sería posible si no tuviéramos una serie de
tecnologías que hagan posible la experiencia de la Web Contextual.

\subsection{Microformatos}

Para la gran mayoría de los usuarios, los microformatos pasan
inadvertidos. Los microformatos conciernen más a los
desarrolladores web que no a los usuarios. En palabras de
\emph{microformats.org}%
\footnote{Jacoby, J. (1984),
\emph{\href{http://www.jstor.org/pss/2488912}{Perspectives on Information Overload}}}

\begin{quote}
diseñados primero para los humanos y segundo para las máquinas, los
microformatos son un formato abierto y simple elaborados a partir
de estándares (\ldots{}) los microformatos intentan resolver
primero problemas simple tomando como referencia conductas y
patrones de uso actuales (ejemplo, XHTML blogging)

\end{quote}
La estructura de los microformatos es simple. Son bloques de HTML
que describen un patrón de información. La ficha personal o de una
organización, un evento, una fecha de calendario, una licencia, la
geolocalización de un lugar, el comentario u opinión de un producto
o etiquetas asociadas a un elemento son ejemplos de microformatos.
Por ejemplo, la siguiente estructura HTML

\begin{verbatim}
<a href="http://creativecommons.org/licenses/by/2.0/" rel="license">cc by 2.0</a>
\end{verbatim}
incluye el atributo \verb!rel="license"! e indica la licencia de la
página.

Seguir esta y las otras convenciones que proponen las diferentes
especificaciones del proyecto facilitará el acceso y el intercambio
de información entre diferentes páginas web. De alguna manera, los
microformatos proporcionan un API casual de la pàgina, una puerta
abierta que permite el acceso sin problemas a los datos que
contiene la página como una interpretación correcta del significado
del contenido por parte de las \emph{máquinas}.

\section{Widgets}

En esencia, los \emph{widgets} son pequeñas aplicaciones
incrustadas en una página. El \emph{widget}, generalmente,
amplifica la información relacionada con el contenido de la página
y evita al visitante buscar esta información fuera de la misma. Los
\emph{widgets} son presentes desde el 2006 en una gran variedad de
formas: relojes, el tiempo, un lector rss específico, información
de servicios de terceros, galerías de fotos, etc.

El 2007 representó la adopción popular de \emph{widgets}%
\footnote{Schneider, Ursula. (2002).
\emph{\href{http://www.jucs.org/jucs_8_5/the_knowledge_attention_gap/Schneider_U.html}{The Knowledge-Attention-Gap: Do We Underestimate The Problem Of Information Overload In Knowledge Management?}}}.
Su presencia comenzó a resultar ubicua en la web. Quizá el widget
más popular y más extendido por su utilización \emph{ad infinitum}
ha resultado ser el de video, especialmente importante en el caso
más que documentado de YouTube y que jugó un papel crucial en su
meteórica aceptación popular. Si en aquel momento Lenehan
argumentaba el papel positivo del uso de los \emph{widgets} por
añadir nuevas funcionalidades a la página.

Si el \emph{widget} proporciona una información específica que, de
alguna manera, evita que el visitante abra una nueva página, y por
consiguiente, limita el efecto \emph{cluttering} de pestañas, en el
otro lado, manifiestan un doble problema. Utilizando la metáfora de
la ciudad, los \emph{widgets} corresponderían en muchas de las
situaciones a las vallas publicitarias; aumentan el ruido en exceso
y proporcionan una experiencia negativa de la lectura de la página.
Se suma a este último que la mayoría de los widgets no estan
personalizados y no tengan ninguna relación temática o visual entre
ellos ni con estilo visual de la página.

En un sentido estricto de la web contextual, los \emph{widgets} no
deberían ser explícitos, como los ejemplos descritos más arriba.
Todo lo contrario, los \emph{widgets} deberían aparecer según la
necesidad del usuario. En esta dirección, uno de los \emph{widgets}
del tipo contextual que apareció con enorme fuerza, especialmente
en el ecosistema de los blogs personales, fue la tecnología
SnapShots%
\footnote{Barthes, Roland. 1980. \emph{La cámara lúcida}. Paidós}.
El lema sigue siendo enriquecer la experiencia informativa del
usuario cuando visite la página. ¿Hacemos referencia a un personaje
histórico? enlazamos aquel con su entrada en la Wikipedia ¿la calle
de un restaurante? ubicamos su posición en una mapa ¿citamos un
monumento? mostramos una galería de fotografías realizadas por los
usuarios. Como ya hemos mencionado arriba, el problema sigue siendo
la saturación del servicio, luego la navegación por la página es
impracticable y resulta enormemente molesta para los usuarios que
la visitan. Exceso de información relacionada con los contenidos.
Esta sobredimensión de información asociada a algunos de los
contenidos de la página ha servido de inspiración para nuevos
servicios. Apture%
\footnote{Italo Calvino, \emph{Las ciudades invisibles}}
proporciona una mejor experiencia de navegación diferente a
SnapShots y viene completada con distintivos visuales que
identifican el carácter de la información asociada (geográfica,
textual, imagen, video, sonido,\ldots{}). En palabras de MacManus%
\footnote{\href{http://en.wikipedia.org/wiki/Personal_information_management}{http://en.wikipedia.org/wiki/Personal\_information\_management}},
la tecnología de Apture se diferencia de tecnologías como Snap en
la que prevalece el sentido común frente a complejos algoritmos de
automatización de la selección de aquellos contenidos suscpetibles
de estar sujetos a un \emph{widget} contextual, caso como Snap o
Adapative SmartLinks%
\footnote{minid.net (2009). \emph{En casa de herrero, cuchillo de palo}
\href{http://www.minid.net/2009/06/17/en-casa-de-herrero-cuchillo-de-palo/}{http://www.minid.net/2009/06/17/en-casa-de-herrero-cuchillo-de-palo/}}.
El futuro de los \emph{widgets} pasará quizá por ser menos
intrusivos, menos \emph{automatizados} y más personales en cuanto
sea el usuario el último en decidir su funcionamiento.

\section{Complementos del navegador}

En el 2003 MacManus%
\footnote{\href{http://googleblog.blogspot.com/2009/05/more-search-options-and-other-updates.html}{More Search Options and other updates from our Searchology event, 2009}}
lamentaba que la visión de la Web Semántica que tenía el padre de
la \emph{World Wide Web}, Sir Tim Berns Lee, en 1993, diez años
después seguía siendo incomprendida o, en el peor de los casos,
indiferente a los caprichos de aquel momento. A principios del
2000, la web mantenía firme la metáfora de una suerte de páginas
amarillas digital, es decir, un mostrador de información,
exclusivamente de lectura y con pocas posibilidades de
participación del usuario. Todo lo contrario del sueño de Tim Berns
Lee, la idea de un navegador/editor de la web en donde la
navegación por la red es sólo una parte de la actividad, no la
única. La cuestión crucial es la capacidad o libertad de
\emph{tejer} la red a través de la escritura por parte de los
usuarios. El navegador deja atrás su metáfora como vehículo de
transporte a través de la \emph{galaxia internet} y adquiere la
capacidad de editor.

\begin{quote}
la descentralización es el principio de diseño subyacente que dará
a la Web Semántica la capacidad de convertirse en algo más que la
suma de sus partes%
\footnote{Mora, Vicente Luis. (2009)}

\end{quote}
Hasta el 2004, el papel del navegador resultaba ser todavía un
simple intermediario. Hasta parecía molestar el lenguaje propio de
la red, el HTML; el distanciamento entre los diseñadores y
desarrolladores y la W3C parecía insalvable. El navegador dominante
en un 95\% era Internet Explorer 6, del que apenas respetaba un
miserable 35\% de los estándares de la web. El navegador, como
aplicación, no proporcionaba otra característica al margen de la
navegación que guardar los marcadores de las páginas; los
\emph{plugins} permitían visualizar los contenidos de cajas negras
como las animaciones realizadas en la tecnología Flash o ejecutar
las aplicaciones Java. El navegador no había evolucionado mucho
desde Mosaic, el primer navegador web de la historia desarrollado
en 1993 por Sir Tim Berns Lee.

La publicación y promoción de la versión 1.0 del navegador Firefox
el 9 de noviembre de 2004%
\footnote{Jones et al., (2006)},
desarrollado bajo el paraguas de la fundación Mozilla, representó
un punto de inflexión en una situación aparentemente inamovible: el
monopolio del uso de Internet Explorer 6 parecía no tener fin aún
sabiéndose que la experiencia de navegación resultaba ser
enormemente pobre, poco fiable y en absoluto segura. El nuevo
navegador, a diferencia de Internet Explorer, nació bajo licencia
Open Source. Las características técnicas que resaltaba la
fundación Mozilla representaban la antítesis de Internet Explorer.
Daba la impresión que los desarrolladores habían tenido muy en
cuenta el malestar de los usuarios y diseñadores web creyentes de
los estándares del W3C. Entre otras características, el navegador
Mozilla Firefox bloqueaba por defecto las ventanas emergentes
(\emph{Pop-up}); mejoraba la seguridad frente a ataques de
\emph{phising} y \emph{spoofing}; incorporaba la navegación por
diferentes páginas a través de un sencillo sistema de pestañas
(\emph{Tabbed Browsing}), facilitaba el proceso de migración de de
migrar los datos como los marcadores. Y por último, era
\emph{extensible}, es decir, su estructura permitía a los
desarrolladores a sumar nuevas funcionalidades al navegador a
través de un simple sistema de complementos (\emph{add-ons}). A
finales de 2004, Firefox contaba con alrededor de más de 100
complementos. Según la Wikipedia, a febrero de 2008 se cuenta
alrededor de 4600 complementos disponibles en diferentes
categorías. La página oficial dedicada%
\footnote{\href{http://www.apple.com/es/mobileme/}{http://www.apple.com/es/mobileme/}}
exclusivamente a la publicación y gestión de los complementos
cuenta con un sistema perfectamente organizado que el resto de
navegadores han ido sumando a su propio ecositema: es el caso de
Internet Explorer a partir de la versión 7%
\footnote{\href{http://www.pogoplug.com/}{http://www.pogoplug.com/}};
de Chrome, el navegador de Google todavía carece de un espacio
dedicado (se espera a partir de mayo de este año dar una respuesta
material); Opera cuenta con \emph{widgets}%
\footnote{\href{https://services.mozilla.com/}{https://services.mozilla.com/}}
y Safari, el navegador de Apple, cuenta también con sus propios%
\footnote{\href{http://www.getdropbox.com/}{http://www.getdropbox.com/}}.
Este interés por ofrecer una versión \emph{enriquecida} del
navegador demuestra que, el papel invisible del navegador hasta
mediados del 2005 ha desaparecido a favor de un mayor
protagonismo.

Los complementos vienen a extender las funcionalidades del
navegador y, por lo tanto, mejorar la experiencia de navegación. La
novedad incluye la actuar como bisagra entre los servicios de la
coversación y el navegador. Complementos enormemente populares como
Shareolic%
\footnote{\href{https://ubuntuone.com/}{https://ubuntuone.com/}}
o Twitterfox%
\footnote{\href{http://desktop.google.com/}{http://desktop.google.com/}}
permiten al usuario publicar, leer y compartir información
múltiples servicios -como recibir notificaciones- en y de la
conversación sin estar obligado a modificar la página activa del
navegador. Este efecto bisagra es posible gracias a las API
abiertas de los servicios de la conversación; la interacción con el
servicio alcanza un horizonte en el que la página web de aquel deja
de \emph{existir} por el hecho mismo de la descentralización. Se
convierten en \emph{bases de datos} abiertos a los usuarios y hacen
posible que aquellos decidan cómo visualizar y gestionar estos
datos.

Los complementos transforman, manipulan, personalizan y aumentan la
experiencia de la navegación. Si cada complemento tiene un fin
específico, la oportunidad de las API abiertas han dibujado el
camino de un complemento que viene a mezclar lo viejo con lo nuevo,
la línea de comandos con los \emph{mashups}. Un complemento que
viene a matar quizá navegación por pestañas.

\subsection{Ubiquity, el \emph{killer-add-on} de Firefox.}

¿Por qué merece una sección aparte Ubiquity? Ubiquity%
\footnote{\href{http://www.microsoft.com/windows/products/winfamily/desktopsearch/default.mspx}{http://www.microsoft.com/windows/products/winfamily/desktopsearch/default.mspx}}
es un proyecto de laboratorio nacido bajo el paraguas de la
fundación Mozilla. Su objetivo es ayudar a los usuarios a reducir
la repetición de pequeñas tareas mientras navegan a través de una
línea de comandos o \emph{verbos}. Tareas cortas como consultar la
ubicación de un lugar en el mapa, la definición de un término, la
entrada de la Wikipedia, la traducción de un fragmento de la
página, etc. nos obliga, en el curso habitual de la navegación, a
abrir continuamente una ventana o pestaña nueva del navegador. Por
el contrario, Ubiquity salva el escollo de la ayuda contextual
convertida en un absoluto \emph{cluttering} de pestañas o ventanas
en la forma de una suerte de capa intermedia entre la estructura de
la página y el usuario.

Este enfoque de una navegación enriquecida a través del uso de
acciones puede representar una puerta abierta a la desaparición de
la organización de la navegación por pestañas. El hijo pequeño de
Ubiquity, Taskfox%
\footnote{\href{http://lifehacker.com/339474/top-10-obscure-google-search-tricks}{http://lifehacker.com/339474/top--10-obscure-google-search-tricks}},
si bien dista del padre de ser un sistema complejo de verbos que
alcance incluso el lenguaje natural, sí persigue una de las
finalidades principales

\begin{quote}
Its aim is to allow users to quickly access information and perform
tasks that would normally take several steps to complete.

\end{quote}
Es interesante comprobar que, finalmente, la severa afirmación de
Oliver Reichenstein sobre la desaparición de las pestañas pueda
venir justamente del uso intensivo de complementos como Ubiquity o
Taskfox, complementos que permitan acceder a la información de la
web de manera asíncrona y a través de una interfaz textual y no
icónica. Por otro lado, esta evolución no sería posible si la
interpretación de la web como

\begin{quote}
a collection of data which can be remixed, mashed together, and
edited by users as well as by web developers%
\footnote{\href{http://mail.google.com/support/bin/answer.py?hl=en&answer=7190}{http://mail.google.com/support/bin/answer.py?hl=en\&answer=7190}}

\end{quote}
es decir, una inconmensurable base de datos en el que todos
participemos, sin exclusión ni distinción, de su creación, edición
y remezcla. Sólo de esta manera, los navegadores podrán dejar atrás
su vieja piel y convertirse en el principal y único \emph{software}
de cualquier dispositivo.

\chap{The sexy job}

\begin{quote}
The purpose of visualization is insight, not pictures[\^{}0]

\end{quote}
De un tiempo a esta parte, la disciplina de la visualización de
datos está siendo objeto de una enorme atención rompiendo,
definitivamente, los límites de la esfera universitaria. Comunicar
gráficamente el \emph{overloading} está de moda; construir el
\emph{relato visual} de una información tabular se ha vuelto en una
actividad profesional muy atractiva

\begin{quote}
I keep saying the sexy job in the next ten years will be
statisticians. People think I'm joking, but who would've guessed
that computer engineers would've been the sexy job of the 1990s?%
\footnote{Blair, Ann. (2003).
\emph{Coping with Information Overload in Early Modern Europe}}

\end{quote}
Esta pasión por la visualización de datos no es casual. La actual
situación de una presencia ubicua de información personal y
colectiva, la capacidad individual y colectiva de producir y
recoger información (gracias a una rápida evolución del perfil
\emph{sensorial} de los nuevos dispositos móviles); capacidad de
mezclarla, remezclarla, publicarla y compartirla en la
\emph{conversación}, la iniciativa de entregar los datos en
formatos abiertos por parte de los servicios 2.0 como de algunos
pocos gobiernos occidentales y, finalmente, el desarrollo de
tecnologías libres dedicadas a la visualización e infografía de
datos, convergen en un espacio poliédrico en el que es posible
reconocer una suerte de \emph{revolución invisible y silenciosa}:
la oportunidad como ciudadano de \emph{opinar} (desde la reflexión
a la crítica) desde una perspectiva visual, no en términos
estéticos y sí en clave de comunicación. Se inaugura la cultura del
\emph{open-and-raw-data}, es decir, el acceso a los datos en su
estado crudo, libre de intermediarios, y en formatos no cautivos o
propietarios. Formatos como CSV, XML, JSON o YAML que faciliten el
proceso de intercambio, interoperabilidad, lectura, análisis,
extracción y manipulación de los datos. La visualización, de otra
manera, resultaría imposible.

\begin{quote}
I would like to suggest: sure, make a beautiful website, but first,
give us – all of us – the unadulterated data. We have to ask for
raw data now.%
\footnote{\href{http://desktop.google.com/}{http://desktop.google.com/}}

\end{quote}
El propósito de la visualización es revelar un significado que, de
otra manera, en su representación más cruda, permanece oculta. De
acuerdo con Tufte (2007)[\^{}tufte], la excelencia de la
representación gráfica de datos consiste en que ideas complejas se
comuniquen con claridad, precisión y eficiencia. En realidad, la
visualización trata de responder a la formulación de una o varias
preguntas en relación a la fuente de información. Encontrar la
respuesta dependerá de una metodología secuencial de trabajo. Ben
Fry (2008) distingue las siguientes fases:

\begin{itemize}
\item
  \textbf{Adquisición} de la fuente de información. Varios autores,
  en los que se incluyen Tufte y Manuel Lima coinciden en la
  necesidad de citar la fuente de información en su formato original.
\item
  \textbf{Minería} y filtrado y extracción de los datos que sean
  relevante al propósito u objetivo planteado al principio.
\item
  \textbf{Elección} de un tipo determinado de representación o
  gráfico que facilite la comunicación de la información representada
  y, a un mismo tiempo, susceptible de ofrecer nuevas
  interpretaciones de los datos seleccionados.
\item
  \textbf{Infovis and Human-Computer Interaction (HCI)} - Frente a
  una representación congelada en papel impreso, la interacción
  invita a una manipulación de la respuesta y enriquece la
  experiencia cognitiva del usuario.
\end{itemize}
La complejidad de la visualización de información reside
precisamente en el acierto de cada una de las fases. Una
visualización no responderá correctamente a la pregunta inicial si
la elección de la fuente de información es equivocada. Por el
contrario, la comunicación de la respuesta no será eficaz si la
representación elegida no es la adecuada en relación a los datos
procesados. Un ejemplo muy extendido de visualización que cumple
con acierto su objetivo son las nubes de etiquetas
(\emph{cloud tags}). La mayoría de páginas incluyen este
\emph{widget}. La nube de etiquetas responde a dos necesidades
complementarias:

\begin{itemize}
\item
  \textbf{Permitir} a los usuarios conocer qué etiquetas clasifican
  el mayor número de contenidos de la página web y, por lo tanto, una
  suerte de \emph{zeitgeist} de qué información ocupa el interés de
  la página -adquisición y análisis de la información-.
\item
  \textbf{Ofrecer} una ruta alternativa de navegación y adquisición
  de la información a través de las etiquetas -interacción con la
  representación-.
\end{itemize}
Si seguimos los pasos indicados por Ben Fry -no sin simplificar al
máximo el proceso- podríamos desglosar el funcionamiento las nubes
de etiquetas de, por ejemplo un blog, en:

\begin{itemize}
\item
  adquisición de la fuente de información en el conjunto total de
  artículos del blog.
\item
  de los artículos sólo nos interesará las etiquetas asociadas a cada
  artículo.
\item
  la representación es una secuencia de palabras -cada palabra
  equivale a una etiqueta- en la que el cuerpo de la fuente vendrá
  determinado por la frecuencia de uso de la etiqueta. A un mayor
  intensidad de uso de una etiqueta, mayor será su cuerpo de letra.
  Otra variante de representación de la nube de etiquetas es también
  conocida como \emph{wordle}[\^{}wordle].
\item
  La interacción, como ya se puede ver en la creación de
  \emph{nubes de etiquetas rotacionales 3D}[\^{}cumulus].
\end{itemize}
\section{La visualización de información es sexy}

Aunque todavía se encuentra encerrada en una aureola de
incomprensión para la mayoría de personas o mal entendida al no
otorgar mayor interés que la de una presentación gráfica de los
datos a través de los métodos más que sabidos (el pastel o el
gráfico de barras), lo cierto es que de un tiempo a esta parte, la
visualización de información ha despertado un enorme interés en la
red. Prueba de ello es el número cada vez mayor de voces que hablan
sobre este tópico, pero no menos importante es la aparición de
atractivas herramientas que permiten generar complejas infografías
con relativa facilidad.

\begin{quote}
To democratize visualization we'll need a new generation of
software. With personal and then web-based computing, we've seen it
happen again and again: spreadsheets, desktop publishing, web
multimedia, cloud-based services. Now, across a range of devices as
well as in the cloud, we have the raw technologies to democratize
the visualization --- and collaborative analysis --- of data.%
\footnote{Hurst, Mark. (2007). \emph{Bit Literacy}. Good Experience Press.}

\end{quote}
Estas herramientas pueden ser de propósito general, provenientes de
la creación gráfica procedural como es el caso del enormemente
popular Processing[\^{}processing] o Nodebox[\^{}nodebox], pero
también han surgido aplicaciones web orientadas al usuario final
como, por ejemplo, Many Eyes[\^{}manyeyes], Swivel[\^{}swivel],
Data360%
\footnote{Hallowell, E. M. (2005).
\emph{\href{http://tr.im/wSsS}{Overloaded circuits: Why smart people underperform}}}.
Pero también merece una enorme atención el goteo continuo de
\emph{frameworks}, \emph{toolkits} o librerías realizadas en
javascript, como por ejemplo Raphaël[\^{}raphael], que permiten una
mayor presencia de la práctica de la visualización de datos en la
web.

Processing nació de las manos Ben Fry y Casey Reas en el 2001 con
la finalidad de proporcionar unas librerías gráficas para el diseño
y artistas desde una visión artística:

\begin{quote}
Processing is the first full-featured programming language and
environment to be created by artists for artists%
\footnote{Malhotra, Naresh K. (1984).
\emph{\href{http://www.jstor.org/pss/2488913}{Reflections on the Information Overload Paradigm in Consumer Decision Making}}}

\end{quote}
Processing no es único en su campo y otras soluciones, quizá no tan
populares, han ido ganando aceptación. El caso de NodeBox o
ContextFree[\^{}contextfree] son claros ejemplos de la misma
metodología de desarrollo en la creación de gráficos. Processing y
NodeBox comparten el origen (MIT) como el planteamiento de
``state machine'' máquina de estados, que facilita la comprensión
del lenguaje al profano y le introduce en el arte de programación.

Si los anteriores son aplicaciones de escritorio, Many Eyes%
\footnote{Jacoby, J. (1984),
\emph{\href{http://www.jstor.org/pss/2488912}{Perspectives on Information Overload}}}
es, en palabras de sus desarrolladores, una aplicación web asentada
en la capacidad de la inteligencia visual en encontrar patrones a
partir de una fuente de información. El objetivo es
\emph{democratizar la visualización y proporcionar nuevas herramientas sociales de análisis de la información},
luego sería motivo de debate y de adquisición de nuevos
conocimientos de la información representada. La cuestión estética
de la representación dejaría paso a una percepción intelectual de
la información. Many Eyes establece seis categorías que agrupan
diferentes sistemas de representación según la metodología del
análisis. Esta categorización no es muy diferente de la mencionada
más arriba:

\begin{itemize}
\item
  Relaciones entre puntos de información (Diagramas de redes,
  ScatterPlot y Matrix Chart)
\item
  Comparación entre un conjunto de valores (Gráfico de burbuja
  \emph{Bubble Chart}, Histogramas, Gráfica de barras)
\item
  Registro de alzas y bajas a lo largo de un tiempo (gráfica de
  líneas y de pilas)
\item
  Observación de las partes de un todo (TreeMap, gráfica de pastel
  \emph{Pie Chart})
\item
  Analizador de texto (Tag Cloud, Phrase Net, Wordle)
\item
  Mapas, geolocalización de la información.
\end{itemize}
Este interés en la democratización de la visualización de la
información no ha quedado relegada a un ámbito de laboratorio
universitario. El diario The New York Times[\^{}nyt] impulsó el año
pasado, desde su versión digital, una suerte de laboratorio donde
los lectores pudieran utilizar la herramienta Many Eyes y la API
del diario y, de esta manera, tener la oportunidad de representar
la información relativa a la actualidad. Si los comentarios a los
artículos publicados en la versión digital del diario como la
sección de los lectores aportan una información propia y
complementan la de los periodistas, las representaciones y los
posibles debates alrededor de las mismas pueden sumar a la
información escrita del diario y aportar un enorme valor social.
Tomemos como ejemplo el discurso de inauguración de Barack Obama%
\footnote{Schneider, Ursula. (2002).
\emph{\href{http://www.jucs.org/jucs_8_5/the_knowledge_attention_gap/Schneider_U.html}{The Knowledge-Attention-Gap: Do We Underestimate The Problem Of Information Overload In Knowledge Management?}}}.
La fuente de información es un texto plano, sin ningún formato que
pueda distraer técnicamente el análisis. Los usuarios toman esta
información y, a través de Many Eyes, eligen un método de
representación y, finalmente, publican el resultado. Swivel%
\footnote{Barthes, Roland. 1980. \emph{La cámara lúcida}. Paidós},
otro servicio que mezcla los tres componentes elementales de
trabajo de Many Eyes (explorar y analizar la información a través
de su representación; compartirla y subir la información relevante
y, por lo tanto, susceptible de generar un debate como la creación)
también proporciona prácticas sugerentes, pero limitadas: la
información relacionada con los accidentes aéreos ocasionados dede
1918[\^{}accidentes], nombran el número de accidentes y fallecidos
pero no especifican el lugar, compañía, recorrido, lugar del
accidente, y una categorización por etiquetas que los clasificara
según el problema que originó el accidente (error humano, técnico,
atentado, fortuito,\ldots{}).

Si todavía la popularidad no es excesiva y el conjunto de las
propuestas no ha conseguido un efecto \emph{democratizador} de la
visualización de la información, sí en cambio comprobamos que
representaciones del tipo Wordle o del Cloud Tag, citada más
arriba, han tenido una buena aceptación y acogida, pero en gran
parte a cuestiones puramente estéticas que no provocadoras de un
posible debate sobre la misma información representada. ¿Demasiado
sujetos todavía a una experiencia puramente retiniana al
enfrentarnos a una representación visual de la información?

[\^{}0]: Ben Shneiderman (1999)%
\footnote{Blair, Ann. (2003).
\emph{Coping with Information Overload in Early Modern Europe}}:
Varian, Hal (2009) \emph{how the Web challenges managers}
\href{http://www.mckinseyquarterly.com/Hal_Varian_on_how_the_Web_challenges_managers_2286}{http://www.mckinseyquarterly.com/Hal\_Varian\_on\_how\_the\_Web\_challenges\_managers\_2286}%
\footnote{Wilson, (2005).
\emph{\href{http://www.newscientist.com/article/mg18624973.400}{Info-overload harms concentration more than marijuana}}}:
Laumans, Joel (2009) \emph{Introduction to Visualizing Data}
\href{http://piksels.com/wp-content/uploads/2009/01/visualizingdata.pdf}{http://piksels.com/wp-content/uploads/2009/01/visualizingdata.pdf}%
\footnote{\href{http://desktop.google.com/}{http://desktop.google.com/}}:
Berners-Lee, Tim, citador por Artur, Charles (2009)
\emph{Web inventor to help Downing Street open up government data}
\href{http://www.guardian.co.uk/technology/2009/jun/10/berners-lee-downing-street-web-open}{http://www.guardian.co.uk/technology/2009/jun/10/berners-lee-downing-street-web-open}
[\^{}tufte]: Tufte ?%
\footnote{Hurst, Mark. (2007). \emph{Bit Literacy}. Good Experience Press.}:
Udell, Jon (2009) \emph{Visualization Trends For The Noosphere}
\href{http://www.visitmix.com/articles/Visualization-Trends-For-The-Noosphere}{http://www.visitmix.com/articles/Visualization-Trends-For-The-Noosphere}%
\footnote{Hallowell, E. M. (2005).
\emph{\href{http://tr.im/wSsS}{Overloaded circuits: Why smart people underperform}}}:
Yau, Nathan (2009) \emph{The Flowing Data guide to visualisations}
\href{http://www.guardian.co.uk/news/datablog/2009/jun/15/google-ibm}{http://www.guardian.co.uk/news/datablog/2009/jun/15/google-ibm}%
\footnote{Malhotra, Naresh K. (1984).
\emph{\href{http://www.jstor.org/pss/2488913}{Reflections on the Information Overload Paradigm in Consumer Decision Making}}}:
Greenberg, Ira (2007).
\emph{Processing: Creative Coding and Computational Art}. Apress%
\footnote{Jacoby, J. (1984),
\emph{\href{http://www.jstor.org/pss/2488912}{Perspectives on Information Overload}}}:
\href{http://manyeyes.alphaworks.ibm.com/manyeyes/page/About.html}{http://manyeyes.alphaworks.ibm.com/manyeyes/page/About.html}%
\footnote{Schneider, Ursula. (2002).
\emph{\href{http://www.jucs.org/jucs_8_5/the_knowledge_attention_gap/Schneider_U.html}{The Knowledge-Attention-Gap: Do We Underestimate The Problem Of Information Overload In Knowledge Management?}}}:
\href{http://vizlab.nytimes.com/datasets/barack-obamas-inauguration-address-2/versions/1}{http://vizlab.nytimes.com/datasets/barack-obamas-inauguration-address--2/versions/1}%
\footnote{Barthes, Roland. 1980. \emph{La cámara lúcida}. Paidós}:
\href{http://www.swivel.com/}{http://www.swivel.com/}
[\^{}accidentes]:
\href{http://www.swivel.com/data_sets/spreadsheet/1018156}{http://www.swivel.com/data\_sets/spreadsheet/1018156}
[\^{}cumulus]:
\href{http://wordpress.org/extend/plugins/wp-cumulus/}{http://wordpress.org/extend/plugins/wp-cumulus/}
[\^{}wordle]: \href{http://www.wordle.net/}{http://www.wordle.net/}
[\^{}processing]:
\href{http://processing.org/}{http://processing.org/}
[\textsuperscript{nodebox]:\href{http://nodebox.net/code/index.php/Home}{http://nodebox.net/code/index.php/Home} [}manyeyes]:
\href{http://manyeyes.alphaworks.ibm.com/manyeyes/}{http://manyeyes.alphaworks.ibm.com/manyeyes/}
[\textsuperscript{swivel]:\href{http://www.swivel.com/}{http://www.swivel.com/} [}raphael]:
\href{http://raphaeljs.com/}{http://raphaeljs.com/}
[\^{}contextfree]:
\href{http://www.contextfreeart.org/}{http://www.contextfreeart.org/}
[\^{}nyt]:
\href{http://vizlab.nytimes.com/}{http://vizlab.nytimes.com/}


\end{document}
